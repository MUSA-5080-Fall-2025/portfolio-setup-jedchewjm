% Options for packages loaded elsewhere
% Options for packages loaded elsewhere
\PassOptionsToPackage{unicode}{hyperref}
\PassOptionsToPackage{hyphens}{url}
\PassOptionsToPackage{dvipsnames,svgnames,x11names}{xcolor}
%
\documentclass[
  letterpaper,
  DIV=11,
  numbers=noendperiod]{scrartcl}
\usepackage{xcolor}
\usepackage{amsmath,amssymb}
\setcounter{secnumdepth}{5}
\usepackage{iftex}
\ifPDFTeX
  \usepackage[T1]{fontenc}
  \usepackage[utf8]{inputenc}
  \usepackage{textcomp} % provide euro and other symbols
\else % if luatex or xetex
  \usepackage{unicode-math} % this also loads fontspec
  \defaultfontfeatures{Scale=MatchLowercase}
  \defaultfontfeatures[\rmfamily]{Ligatures=TeX,Scale=1}
\fi
\usepackage{lmodern}
\ifPDFTeX\else
  % xetex/luatex font selection
\fi
% Use upquote if available, for straight quotes in verbatim environments
\IfFileExists{upquote.sty}{\usepackage{upquote}}{}
\IfFileExists{microtype.sty}{% use microtype if available
  \usepackage[]{microtype}
  \UseMicrotypeSet[protrusion]{basicmath} % disable protrusion for tt fonts
}{}
\makeatletter
\@ifundefined{KOMAClassName}{% if non-KOMA class
  \IfFileExists{parskip.sty}{%
    \usepackage{parskip}
  }{% else
    \setlength{\parindent}{0pt}
    \setlength{\parskip}{6pt plus 2pt minus 1pt}}
}{% if KOMA class
  \KOMAoptions{parskip=half}}
\makeatother
% Make \paragraph and \subparagraph free-standing
\makeatletter
\ifx\paragraph\undefined\else
  \let\oldparagraph\paragraph
  \renewcommand{\paragraph}{
    \@ifstar
      \xxxParagraphStar
      \xxxParagraphNoStar
  }
  \newcommand{\xxxParagraphStar}[1]{\oldparagraph*{#1}\mbox{}}
  \newcommand{\xxxParagraphNoStar}[1]{\oldparagraph{#1}\mbox{}}
\fi
\ifx\subparagraph\undefined\else
  \let\oldsubparagraph\subparagraph
  \renewcommand{\subparagraph}{
    \@ifstar
      \xxxSubParagraphStar
      \xxxSubParagraphNoStar
  }
  \newcommand{\xxxSubParagraphStar}[1]{\oldsubparagraph*{#1}\mbox{}}
  \newcommand{\xxxSubParagraphNoStar}[1]{\oldsubparagraph{#1}\mbox{}}
\fi
\makeatother

\usepackage{color}
\usepackage{fancyvrb}
\newcommand{\VerbBar}{|}
\newcommand{\VERB}{\Verb[commandchars=\\\{\}]}
\DefineVerbatimEnvironment{Highlighting}{Verbatim}{commandchars=\\\{\}}
% Add ',fontsize=\small' for more characters per line
\usepackage{framed}
\definecolor{shadecolor}{RGB}{241,243,245}
\newenvironment{Shaded}{\begin{snugshade}}{\end{snugshade}}
\newcommand{\AlertTok}[1]{\textcolor[rgb]{0.68,0.00,0.00}{#1}}
\newcommand{\AnnotationTok}[1]{\textcolor[rgb]{0.37,0.37,0.37}{#1}}
\newcommand{\AttributeTok}[1]{\textcolor[rgb]{0.40,0.45,0.13}{#1}}
\newcommand{\BaseNTok}[1]{\textcolor[rgb]{0.68,0.00,0.00}{#1}}
\newcommand{\BuiltInTok}[1]{\textcolor[rgb]{0.00,0.23,0.31}{#1}}
\newcommand{\CharTok}[1]{\textcolor[rgb]{0.13,0.47,0.30}{#1}}
\newcommand{\CommentTok}[1]{\textcolor[rgb]{0.37,0.37,0.37}{#1}}
\newcommand{\CommentVarTok}[1]{\textcolor[rgb]{0.37,0.37,0.37}{\textit{#1}}}
\newcommand{\ConstantTok}[1]{\textcolor[rgb]{0.56,0.35,0.01}{#1}}
\newcommand{\ControlFlowTok}[1]{\textcolor[rgb]{0.00,0.23,0.31}{\textbf{#1}}}
\newcommand{\DataTypeTok}[1]{\textcolor[rgb]{0.68,0.00,0.00}{#1}}
\newcommand{\DecValTok}[1]{\textcolor[rgb]{0.68,0.00,0.00}{#1}}
\newcommand{\DocumentationTok}[1]{\textcolor[rgb]{0.37,0.37,0.37}{\textit{#1}}}
\newcommand{\ErrorTok}[1]{\textcolor[rgb]{0.68,0.00,0.00}{#1}}
\newcommand{\ExtensionTok}[1]{\textcolor[rgb]{0.00,0.23,0.31}{#1}}
\newcommand{\FloatTok}[1]{\textcolor[rgb]{0.68,0.00,0.00}{#1}}
\newcommand{\FunctionTok}[1]{\textcolor[rgb]{0.28,0.35,0.67}{#1}}
\newcommand{\ImportTok}[1]{\textcolor[rgb]{0.00,0.46,0.62}{#1}}
\newcommand{\InformationTok}[1]{\textcolor[rgb]{0.37,0.37,0.37}{#1}}
\newcommand{\KeywordTok}[1]{\textcolor[rgb]{0.00,0.23,0.31}{\textbf{#1}}}
\newcommand{\NormalTok}[1]{\textcolor[rgb]{0.00,0.23,0.31}{#1}}
\newcommand{\OperatorTok}[1]{\textcolor[rgb]{0.37,0.37,0.37}{#1}}
\newcommand{\OtherTok}[1]{\textcolor[rgb]{0.00,0.23,0.31}{#1}}
\newcommand{\PreprocessorTok}[1]{\textcolor[rgb]{0.68,0.00,0.00}{#1}}
\newcommand{\RegionMarkerTok}[1]{\textcolor[rgb]{0.00,0.23,0.31}{#1}}
\newcommand{\SpecialCharTok}[1]{\textcolor[rgb]{0.37,0.37,0.37}{#1}}
\newcommand{\SpecialStringTok}[1]{\textcolor[rgb]{0.13,0.47,0.30}{#1}}
\newcommand{\StringTok}[1]{\textcolor[rgb]{0.13,0.47,0.30}{#1}}
\newcommand{\VariableTok}[1]{\textcolor[rgb]{0.07,0.07,0.07}{#1}}
\newcommand{\VerbatimStringTok}[1]{\textcolor[rgb]{0.13,0.47,0.30}{#1}}
\newcommand{\WarningTok}[1]{\textcolor[rgb]{0.37,0.37,0.37}{\textit{#1}}}

\usepackage{longtable,booktabs,array}
\usepackage{calc} % for calculating minipage widths
% Correct order of tables after \paragraph or \subparagraph
\usepackage{etoolbox}
\makeatletter
\patchcmd\longtable{\par}{\if@noskipsec\mbox{}\fi\par}{}{}
\makeatother
% Allow footnotes in longtable head/foot
\IfFileExists{footnotehyper.sty}{\usepackage{footnotehyper}}{\usepackage{footnote}}
\makesavenoteenv{longtable}
\usepackage{graphicx}
\makeatletter
\newsavebox\pandoc@box
\newcommand*\pandocbounded[1]{% scales image to fit in text height/width
  \sbox\pandoc@box{#1}%
  \Gscale@div\@tempa{\textheight}{\dimexpr\ht\pandoc@box+\dp\pandoc@box\relax}%
  \Gscale@div\@tempb{\linewidth}{\wd\pandoc@box}%
  \ifdim\@tempb\p@<\@tempa\p@\let\@tempa\@tempb\fi% select the smaller of both
  \ifdim\@tempa\p@<\p@\scalebox{\@tempa}{\usebox\pandoc@box}%
  \else\usebox{\pandoc@box}%
  \fi%
}
% Set default figure placement to htbp
\def\fps@figure{htbp}
\makeatother





\setlength{\emergencystretch}{3em} % prevent overfull lines

\providecommand{\tightlist}{%
  \setlength{\itemsep}{0pt}\setlength{\parskip}{0pt}}



 


\usepackage{booktabs}
\usepackage{longtable}
\usepackage{array}
\usepackage{multirow}
\usepackage{wrapfig}
\usepackage{float}
\usepackage{colortbl}
\usepackage{pdflscape}
\usepackage{tabu}
\usepackage{threeparttable}
\usepackage{threeparttablex}
\usepackage[normalem]{ulem}
\usepackage{makecell}
\usepackage{xcolor}
\KOMAoption{captions}{tableheading}
\makeatletter
\@ifpackageloaded{tcolorbox}{}{\usepackage[skins,breakable]{tcolorbox}}
\@ifpackageloaded{fontawesome5}{}{\usepackage{fontawesome5}}
\definecolor{quarto-callout-color}{HTML}{909090}
\definecolor{quarto-callout-note-color}{HTML}{0758E5}
\definecolor{quarto-callout-important-color}{HTML}{CC1914}
\definecolor{quarto-callout-warning-color}{HTML}{EB9113}
\definecolor{quarto-callout-tip-color}{HTML}{00A047}
\definecolor{quarto-callout-caution-color}{HTML}{FC5300}
\definecolor{quarto-callout-color-frame}{HTML}{acacac}
\definecolor{quarto-callout-note-color-frame}{HTML}{4582ec}
\definecolor{quarto-callout-important-color-frame}{HTML}{d9534f}
\definecolor{quarto-callout-warning-color-frame}{HTML}{f0ad4e}
\definecolor{quarto-callout-tip-color-frame}{HTML}{02b875}
\definecolor{quarto-callout-caution-color-frame}{HTML}{fd7e14}
\makeatother
\makeatletter
\@ifpackageloaded{caption}{}{\usepackage{caption}}
\AtBeginDocument{%
\ifdefined\contentsname
  \renewcommand*\contentsname{Table of contents}
\else
  \newcommand\contentsname{Table of contents}
\fi
\ifdefined\listfigurename
  \renewcommand*\listfigurename{List of Figures}
\else
  \newcommand\listfigurename{List of Figures}
\fi
\ifdefined\listtablename
  \renewcommand*\listtablename{List of Tables}
\else
  \newcommand\listtablename{List of Tables}
\fi
\ifdefined\figurename
  \renewcommand*\figurename{Figure}
\else
  \newcommand\figurename{Figure}
\fi
\ifdefined\tablename
  \renewcommand*\tablename{Table}
\else
  \newcommand\tablename{Table}
\fi
}
\@ifpackageloaded{float}{}{\usepackage{float}}
\floatstyle{ruled}
\@ifundefined{c@chapter}{\newfloat{codelisting}{h}{lop}}{\newfloat{codelisting}{h}{lop}[chapter]}
\floatname{codelisting}{Listing}
\newcommand*\listoflistings{\listof{codelisting}{List of Listings}}
\makeatother
\makeatletter
\makeatother
\makeatletter
\@ifpackageloaded{caption}{}{\usepackage{caption}}
\@ifpackageloaded{subcaption}{}{\usepackage{subcaption}}
\makeatother
\usepackage{bookmark}
\IfFileExists{xurl.sty}{\usepackage{xurl}}{} % add URL line breaks if available
\urlstyle{same}
\hypersetup{
  pdftitle={Assignment 4: Predictive Policing \& Analysis},
  pdfauthor={Jed Chew},
  colorlinks=true,
  linkcolor={blue},
  filecolor={Maroon},
  citecolor={Blue},
  urlcolor={Blue},
  pdfcreator={LaTeX via pandoc}}


\title{Assignment 4: Predictive Policing \& Analysis}
\author{Jed Chew}
\date{2025-11-10}
\begin{document}
\maketitle

\renewcommand*\contentsname{Table of contents}
{
\hypersetup{linkcolor=}
\setcounter{tocdepth}{3}
\tableofcontents
}

\subsection{About This Exercise}\label{about-this-exercise}

In this lab, you will apply the spatial predictive modeling techniques
demonstrated in the class exercise to a 311 service request type of your
choice. You will build a complete spatial predictive model, document
your process, and interpret your results.

\section{Learning Objectives}\label{learning-objectives}

By the end of this exercise, you will be able to:

\begin{enumerate}
\def\labelenumi{\arabic{enumi}.}
\tightlist
\item
  Create a fishnet grid for aggregating point-level crime data
\item
  Calculate spatial features including k-nearest neighbors and distance
  measures
\item
  Diagnose spatial autocorrelation using Local Moran's I
\item
  Fit and interpret Poisson and Negative Binomial regression for count
  data
\item
  Implement spatial cross-validation (Leave-One-Group-Out)
\item
  Compare model predictions to a Kernel Density Estimation baseline
\item
  Evaluate model performance using appropriate metrics
\end{enumerate}

\section{Setup}\label{setup}

\begin{Shaded}
\begin{Highlighting}[]
\CommentTok{\# Load required packages}
\FunctionTok{library}\NormalTok{(tidyverse)      }\CommentTok{\# Data manipulation}
\FunctionTok{library}\NormalTok{(sf)             }\CommentTok{\# Spatial operations}
\FunctionTok{library}\NormalTok{(here)           }\CommentTok{\# Relative file paths}
\FunctionTok{library}\NormalTok{(viridis)        }\CommentTok{\# Color scales}
\FunctionTok{library}\NormalTok{(terra)          }\CommentTok{\# Raster operations (replaces \textquotesingle{}raster\textquotesingle{})}
\FunctionTok{library}\NormalTok{(spdep)          }\CommentTok{\# Spatial dependence}
\FunctionTok{library}\NormalTok{(FNN)            }\CommentTok{\# Fast nearest neighbors}
\FunctionTok{library}\NormalTok{(MASS)           }\CommentTok{\# Negative binomial regression}
\FunctionTok{library}\NormalTok{(patchwork)      }\CommentTok{\# Plot composition (replaces grid/gridExtra)}
\FunctionTok{library}\NormalTok{(knitr)          }\CommentTok{\# Tables}
\FunctionTok{library}\NormalTok{(kableExtra)     }\CommentTok{\# Table formatting}
\FunctionTok{library}\NormalTok{(classInt)       }\CommentTok{\# Classification intervals}
\FunctionTok{library}\NormalTok{(here)}

\CommentTok{\# Spatstat split into sub{-}packages}
\FunctionTok{library}\NormalTok{(spatstat.geom)    }\CommentTok{\# Spatial geometries}
\FunctionTok{library}\NormalTok{(spatstat.explore) }\CommentTok{\# Spatial exploration/KDE}

\CommentTok{\# Set options}
\FunctionTok{options}\NormalTok{(}\AttributeTok{scipen =} \DecValTok{999}\NormalTok{)  }\CommentTok{\# No scientific notation}
\FunctionTok{set.seed}\NormalTok{(}\DecValTok{5080}\NormalTok{)         }\CommentTok{\# Reproducibility}

\CommentTok{\# Create consistent theme for visualizations}
\NormalTok{theme\_crime }\OtherTok{\textless{}{-}} \ControlFlowTok{function}\NormalTok{(}\AttributeTok{base\_size =} \DecValTok{11}\NormalTok{) \{}
  \FunctionTok{theme\_minimal}\NormalTok{(}\AttributeTok{base\_size =}\NormalTok{ base\_size) }\SpecialCharTok{+}
    \FunctionTok{theme}\NormalTok{(}
      \AttributeTok{plot.title =} \FunctionTok{element\_text}\NormalTok{(}\AttributeTok{face =} \StringTok{"bold"}\NormalTok{, }\AttributeTok{size =}\NormalTok{ base\_size }\SpecialCharTok{+} \DecValTok{1}\NormalTok{),}
      \AttributeTok{plot.subtitle =} \FunctionTok{element\_text}\NormalTok{(}\AttributeTok{color =} \StringTok{"gray30"}\NormalTok{, }\AttributeTok{size =}\NormalTok{ base\_size }\SpecialCharTok{{-}} \DecValTok{1}\NormalTok{),}
      \AttributeTok{legend.position =} \StringTok{"right"}\NormalTok{,}
      \AttributeTok{panel.grid.minor =} \FunctionTok{element\_blank}\NormalTok{(),}
      \AttributeTok{axis.text =} \FunctionTok{element\_blank}\NormalTok{(),}
      \AttributeTok{axis.title =} \FunctionTok{element\_blank}\NormalTok{()}
\NormalTok{    )}
\NormalTok{\}}

\CommentTok{\# Set as default}
\FunctionTok{theme\_set}\NormalTok{(}\FunctionTok{theme\_crime}\NormalTok{())}

\CommentTok{\# Set Working Directory}
\FunctionTok{setwd}\NormalTok{(}\StringTok{"C:/Users/chewj/Documents/MUSA/Github/portfolio{-}setup{-}jedchewjm/labs/lab\_4"}\NormalTok{)}

\FunctionTok{cat}\NormalTok{(}\StringTok{"✓ All packages loaded successfully!}\SpecialCharTok{\textbackslash{}n}\StringTok{"}\NormalTok{)}
\end{Highlighting}
\end{Shaded}

\begin{verbatim}
✓ All packages loaded successfully!
\end{verbatim}

\begin{Shaded}
\begin{Highlighting}[]
\FunctionTok{cat}\NormalTok{(}\StringTok{"✓ Working directory:"}\NormalTok{, }\FunctionTok{getwd}\NormalTok{(), }\StringTok{"}\SpecialCharTok{\textbackslash{}n}\StringTok{"}\NormalTok{)}
\end{Highlighting}
\end{Shaded}

\begin{verbatim}
✓ Working directory: C:/Users/chewj/Documents/MUSA/Github/portfolio-setup-jedchewjm/labs/lab_4 
\end{verbatim}

\section{Part 1: Analyze Chosen 311 Violation
Type}\label{part-1-analyze-chosen-311-violation-type}

\section{Choose your 311 Violation
Type}\label{choose-your-311-violation-type}

\begin{itemize}
\item
  Browse the available service request types (e.g., Graffiti Removal,
  Pothole Repair, Street Light Out, Sanitation Code Violations, etc.)
  and \textbf{choose one violation type} that interests you.
\item
  Chicago 311 Service Requests dataset:
  \href{https://data.cityofchicago.org/stories/s/311-Dataset-Changes-12-11-2018/d7nq-5g7t}{\textbf{https://data.cityofchicago.org/stories/s/311-Dataset-Changes-12-11-2018/d7nq-5g7t}}
\item
  Use \texttt{ESRI:102271} (Illinois State Plane East, NAD83, US Feet)
  as CRS
\end{itemize}

\subsubsection{1.1 \textbar{} Load Chicago Spatial
Data}\label{load-chicago-spatial-data}

\begin{Shaded}
\begin{Highlighting}[]
\CommentTok{\# Load police districts (used for spatial cross{-}validation)}
\NormalTok{policeDistricts }\OtherTok{\textless{}{-}} 
  \FunctionTok{st\_read}\NormalTok{(}\StringTok{"https://data.cityofchicago.org/api/geospatial/24zt{-}jpfn?method=export\&format=GeoJSON"}\NormalTok{) }\SpecialCharTok{\%\textgreater{}\%}
  \FunctionTok{st\_transform}\NormalTok{(}\StringTok{\textquotesingle{}ESRI:102271\textquotesingle{}}\NormalTok{) }\SpecialCharTok{\%\textgreater{}\%}
\NormalTok{  dplyr}\SpecialCharTok{::}\FunctionTok{select}\NormalTok{(}\AttributeTok{District =}\NormalTok{ dist\_num)}
\end{Highlighting}
\end{Shaded}

\begin{verbatim}
Reading layer `OGRGeoJSON' from data source 
  `https://data.cityofchicago.org/api/geospatial/24zt-jpfn?method=export&format=GeoJSON' 
  using driver `GeoJSON'
Simple feature collection with 25 features and 2 fields
Geometry type: MULTIPOLYGON
Dimension:     XY
Bounding box:  xmin: -87.94011 ymin: 41.64455 xmax: -87.52414 ymax: 42.02303
Geodetic CRS:  WGS 84
\end{verbatim}

\begin{Shaded}
\begin{Highlighting}[]
\CommentTok{\# Load police beats (smaller administrative units)}
\NormalTok{policeBeats }\OtherTok{\textless{}{-}} 
  \FunctionTok{st\_read}\NormalTok{(}\StringTok{"https://data.cityofchicago.org/api/geospatial/n9it{-}hstw?method=export\&format=GeoJSON"}\NormalTok{) }\SpecialCharTok{\%\textgreater{}\%}
  \FunctionTok{st\_transform}\NormalTok{(}\StringTok{\textquotesingle{}ESRI:102271\textquotesingle{}}\NormalTok{) }\SpecialCharTok{\%\textgreater{}\%}
\NormalTok{  dplyr}\SpecialCharTok{::}\FunctionTok{select}\NormalTok{(}\AttributeTok{Beat =}\NormalTok{ beat\_num)}
\end{Highlighting}
\end{Shaded}

\begin{verbatim}
Reading layer `OGRGeoJSON' from data source 
  `https://data.cityofchicago.org/api/geospatial/n9it-hstw?method=export&format=GeoJSON' 
  using driver `GeoJSON'
Simple feature collection with 277 features and 4 fields
Geometry type: MULTIPOLYGON
Dimension:     XY
Bounding box:  xmin: -87.94011 ymin: 41.64455 xmax: -87.52414 ymax: 42.02303
Geodetic CRS:  WGS 84
\end{verbatim}

\begin{Shaded}
\begin{Highlighting}[]
\CommentTok{\# Load Chicago boundary}
\NormalTok{chicagoBoundary }\OtherTok{\textless{}{-}} 
  \FunctionTok{st\_read}\NormalTok{(}\StringTok{"https://raw.githubusercontent.com/urbanSpatial/Public{-}Policy{-}Analytics{-}Landing/master/DATA/Chapter5/chicagoBoundary.geojson"}\NormalTok{) }\SpecialCharTok{\%\textgreater{}\%}
  \FunctionTok{st\_transform}\NormalTok{(}\StringTok{\textquotesingle{}ESRI:102271\textquotesingle{}}\NormalTok{)}
\end{Highlighting}
\end{Shaded}

\begin{verbatim}
Reading layer `chicagoBoundary' from data source 
  `https://raw.githubusercontent.com/urbanSpatial/Public-Policy-Analytics-Landing/master/DATA/Chapter5/chicagoBoundary.geojson' 
  using driver `GeoJSON'
Simple feature collection with 1 feature and 1 field
Geometry type: POLYGON
Dimension:     XY
Bounding box:  xmin: -87.8367 ymin: 41.64454 xmax: -87.52414 ymax: 42.02304
Geodetic CRS:  WGS 84
\end{verbatim}

\begin{Shaded}
\begin{Highlighting}[]
\FunctionTok{cat}\NormalTok{(}\StringTok{"✓ Loaded spatial boundaries}\SpecialCharTok{\textbackslash{}n}\StringTok{"}\NormalTok{)}
\end{Highlighting}
\end{Shaded}

\begin{verbatim}
✓ Loaded spatial boundaries
\end{verbatim}

\begin{Shaded}
\begin{Highlighting}[]
\FunctionTok{cat}\NormalTok{(}\StringTok{"  {-} Police districts:"}\NormalTok{, }\FunctionTok{nrow}\NormalTok{(policeDistricts), }\StringTok{"}\SpecialCharTok{\textbackslash{}n}\StringTok{"}\NormalTok{)}
\end{Highlighting}
\end{Shaded}

\begin{verbatim}
  - Police districts: 25 
\end{verbatim}

\begin{Shaded}
\begin{Highlighting}[]
\FunctionTok{cat}\NormalTok{(}\StringTok{"  {-} Police beats:"}\NormalTok{, }\FunctionTok{nrow}\NormalTok{(policeBeats), }\StringTok{"}\SpecialCharTok{\textbackslash{}n}\StringTok{"}\NormalTok{)}
\end{Highlighting}
\end{Shaded}

\begin{verbatim}
  - Police beats: 277 
\end{verbatim}

\subsubsection{1.2 \textbar{} Load Burglary
Data}\label{load-burglary-data}

\begin{Shaded}
\begin{Highlighting}[]
\CommentTok{\# Load from provided data file (downloaded from Chicago open data portal)}
\NormalTok{burglaries }\OtherTok{\textless{}{-}} \FunctionTok{st\_read}\NormalTok{(}\StringTok{"data/burglaries.shp"}\NormalTok{) }\SpecialCharTok{\%\textgreater{}\%} 
  \FunctionTok{st\_transform}\NormalTok{(}\StringTok{\textquotesingle{}ESRI:102271\textquotesingle{}}\NormalTok{)}
\end{Highlighting}
\end{Shaded}

\begin{verbatim}
Reading layer `burglaries' from data source 
  `C:\Users\chewj\Documents\MUSA\Github\portfolio-setup-jedchewjm\labs\lab_4\data\burglaries.shp' 
  using driver `ESRI Shapefile'
Simple feature collection with 7482 features and 22 fields
Geometry type: POINT
Dimension:     XY
Bounding box:  xmin: 340492 ymin: 552959.6 xmax: 367153.5 ymax: 594815.1
Projected CRS: NAD83(HARN) / Illinois East
\end{verbatim}

\begin{Shaded}
\begin{Highlighting}[]
\CommentTok{\# Check the data}
\FunctionTok{cat}\NormalTok{(}\StringTok{"}\SpecialCharTok{\textbackslash{}n}\StringTok{✓ Loaded burglary data}\SpecialCharTok{\textbackslash{}n}\StringTok{"}\NormalTok{)}
\end{Highlighting}
\end{Shaded}

\begin{verbatim}

✓ Loaded burglary data
\end{verbatim}

\begin{Shaded}
\begin{Highlighting}[]
\FunctionTok{cat}\NormalTok{(}\StringTok{"  {-} Number of burglaries:"}\NormalTok{, }\FunctionTok{nrow}\NormalTok{(burglaries), }\StringTok{"}\SpecialCharTok{\textbackslash{}n}\StringTok{"}\NormalTok{)}
\end{Highlighting}
\end{Shaded}

\begin{verbatim}
  - Number of burglaries: 7482 
\end{verbatim}

\begin{Shaded}
\begin{Highlighting}[]
\FunctionTok{cat}\NormalTok{(}\StringTok{"  {-} CRS:"}\NormalTok{, }\FunctionTok{st\_crs}\NormalTok{(burglaries)}\SpecialCharTok{$}\NormalTok{input, }\StringTok{"}\SpecialCharTok{\textbackslash{}n}\StringTok{"}\NormalTok{)}
\end{Highlighting}
\end{Shaded}

\begin{verbatim}
  - CRS: ESRI:102271 
\end{verbatim}

\begin{Shaded}
\begin{Highlighting}[]
\FunctionTok{cat}\NormalTok{(}\StringTok{"  {-} Date range:"}\NormalTok{, }\FunctionTok{min}\NormalTok{(burglaries}\SpecialCharTok{$}\NormalTok{date, }\AttributeTok{na.rm =} \ConstantTok{TRUE}\NormalTok{), }\StringTok{"to"}\NormalTok{, }
    \FunctionTok{max}\NormalTok{(burglaries}\SpecialCharTok{$}\NormalTok{date, }\AttributeTok{na.rm =} \ConstantTok{TRUE}\NormalTok{), }\StringTok{"}\SpecialCharTok{\textbackslash{}n}\StringTok{"}\NormalTok{)}
\end{Highlighting}
\end{Shaded}

\begin{verbatim}
  - Date range: Inf to -Inf 
\end{verbatim}

\subsubsection{1.3 \textbar{} Visualize Point Data \& Describe
Patterns}\label{visualize-point-data-describe-patterns}

\begin{Shaded}
\begin{Highlighting}[]
\CommentTok{\# Simple point map}
\NormalTok{p1 }\OtherTok{\textless{}{-}} \FunctionTok{ggplot}\NormalTok{() }\SpecialCharTok{+} 
  \FunctionTok{geom\_sf}\NormalTok{(}\AttributeTok{data =}\NormalTok{ chicagoBoundary, }\AttributeTok{fill =} \StringTok{"gray95"}\NormalTok{, }\AttributeTok{color =} \StringTok{"gray60"}\NormalTok{) }\SpecialCharTok{+}
  \FunctionTok{geom\_sf}\NormalTok{(}\AttributeTok{data =}\NormalTok{ burglaries, }\AttributeTok{color =} \StringTok{"\#d62828"}\NormalTok{, }\AttributeTok{size =} \FloatTok{0.1}\NormalTok{, }\AttributeTok{alpha =} \FloatTok{0.4}\NormalTok{) }\SpecialCharTok{+}
  \FunctionTok{labs}\NormalTok{(}
    \AttributeTok{title =} \StringTok{"Burglary Locations"}\NormalTok{,}
    \AttributeTok{subtitle =} \FunctionTok{paste0}\NormalTok{(}\StringTok{"Chicago 2017, n = "}\NormalTok{, }\FunctionTok{nrow}\NormalTok{(burglaries))}
\NormalTok{  )}

\CommentTok{\# Density surface using modern syntax}
\NormalTok{p2 }\OtherTok{\textless{}{-}} \FunctionTok{ggplot}\NormalTok{() }\SpecialCharTok{+} 
  \FunctionTok{geom\_sf}\NormalTok{(}\AttributeTok{data =}\NormalTok{ chicagoBoundary, }\AttributeTok{fill =} \StringTok{"gray95"}\NormalTok{, }\AttributeTok{color =} \StringTok{"gray60"}\NormalTok{) }\SpecialCharTok{+}
  \FunctionTok{geom\_density\_2d\_filled}\NormalTok{(}
    \AttributeTok{data =} \FunctionTok{data.frame}\NormalTok{(}\FunctionTok{st\_coordinates}\NormalTok{(burglaries)),}
    \FunctionTok{aes}\NormalTok{(X, Y),}
    \AttributeTok{alpha =} \FloatTok{0.7}\NormalTok{,}
    \AttributeTok{bins =} \DecValTok{8}
\NormalTok{  ) }\SpecialCharTok{+}
  \FunctionTok{scale\_fill\_viridis\_d}\NormalTok{(}
    \AttributeTok{option =} \StringTok{"plasma"}\NormalTok{,}
    \AttributeTok{direction =} \SpecialCharTok{{-}}\DecValTok{1}\NormalTok{,}
    \AttributeTok{guide =} \StringTok{"none"}  \CommentTok{\# Modern ggplot2 syntax (not guide = FALSE)}
\NormalTok{  ) }\SpecialCharTok{+}
  \FunctionTok{labs}\NormalTok{(}
    \AttributeTok{title =} \StringTok{"Density Surface"}\NormalTok{,}
    \AttributeTok{subtitle =} \StringTok{"Kernel density estimation"}
\NormalTok{  )}

\CommentTok{\# Combine plots using patchwork (modern approach)}
\NormalTok{p1 }\SpecialCharTok{+}\NormalTok{ p2 }\SpecialCharTok{+} 
  \FunctionTok{plot\_annotation}\NormalTok{(}
    \AttributeTok{title =} \StringTok{"Spatial Distribution of Burglaries in Chicago"}\NormalTok{,}
    \AttributeTok{tag\_levels =} \StringTok{\textquotesingle{}A\textquotesingle{}}
\NormalTok{  )}
\end{Highlighting}
\end{Shaded}

\pandocbounded{\includegraphics[keepaspectratio]{assignment4_template_files/figure-pdf/visualize-points-1.pdf}}

\begin{tcolorbox}[enhanced jigsaw, colback=white, toprule=.15mm, rightrule=.15mm, bottomrule=.15mm, left=2mm, opacitybacktitle=0.6, coltitle=black, colbacktitle=quarto-callout-note-color!10!white, titlerule=0mm, opacityback=0, bottomtitle=1mm, toptitle=1mm, title=\textcolor{quarto-callout-note-color}{\faInfo}\hspace{0.5em}{Observed Patterns}, arc=.35mm, leftrule=.75mm, breakable, colframe=quarto-callout-note-color-frame]

\begin{itemize}
\tightlist
\item
  Lorem Ipsum
\item
  Lorem Ipsum
\end{itemize}

\end{tcolorbox}

\section{Part 2: Create Fishnet Grid}\label{part-2-create-fishnet-grid}

A \textbf{fishnet grid} converts irregular point data into a regular
grid of cells where we can aggregate counts / calculate spatial features
/ apply regression models

\subsubsection{2.1 \textbar{} Create 500 m x 500 m Fishnet
Grid}\label{create-500-m-x-500-m-fishnet-grid}

\begin{Shaded}
\begin{Highlighting}[]
\CommentTok{\# Create 500m x 500m grid}
\NormalTok{fishnet }\OtherTok{\textless{}{-}} \FunctionTok{st\_make\_grid}\NormalTok{(}
\NormalTok{  chicagoBoundary,}
  \AttributeTok{cellsize =} \DecValTok{500}\NormalTok{,  }\CommentTok{\# 500 meters per cell}
  \AttributeTok{square =} \ConstantTok{TRUE}
\NormalTok{) }\SpecialCharTok{\%\textgreater{}\%}
  \FunctionTok{st\_sf}\NormalTok{() }\SpecialCharTok{\%\textgreater{}\%}
  \FunctionTok{mutate}\NormalTok{(}\AttributeTok{uniqueID =} \FunctionTok{row\_number}\NormalTok{())}

\CommentTok{\# Keep only cells that intersect Chicago}
\NormalTok{fishnet }\OtherTok{\textless{}{-}}\NormalTok{ fishnet[chicagoBoundary, ]}

\CommentTok{\# View basic info}
\FunctionTok{cat}\NormalTok{(}\StringTok{"✓ Created fishnet grid}\SpecialCharTok{\textbackslash{}n}\StringTok{"}\NormalTok{)}
\end{Highlighting}
\end{Shaded}

\begin{verbatim}
✓ Created fishnet grid
\end{verbatim}

\begin{Shaded}
\begin{Highlighting}[]
\FunctionTok{cat}\NormalTok{(}\StringTok{"  {-} Number of cells:"}\NormalTok{, }\FunctionTok{nrow}\NormalTok{(fishnet), }\StringTok{"}\SpecialCharTok{\textbackslash{}n}\StringTok{"}\NormalTok{)}
\end{Highlighting}
\end{Shaded}

\begin{verbatim}
  - Number of cells: 2458 
\end{verbatim}

\begin{Shaded}
\begin{Highlighting}[]
\FunctionTok{cat}\NormalTok{(}\StringTok{"  {-} Cell size:"}\NormalTok{, }\DecValTok{500}\NormalTok{, }\StringTok{"x"}\NormalTok{, }\DecValTok{500}\NormalTok{, }\StringTok{"meters}\SpecialCharTok{\textbackslash{}n}\StringTok{"}\NormalTok{)}
\end{Highlighting}
\end{Shaded}

\begin{verbatim}
  - Cell size: 500 x 500 meters
\end{verbatim}

\begin{Shaded}
\begin{Highlighting}[]
\FunctionTok{cat}\NormalTok{(}\StringTok{"  {-} Cell area:"}\NormalTok{, }\FunctionTok{round}\NormalTok{(}\FunctionTok{st\_area}\NormalTok{(fishnet[}\DecValTok{1}\NormalTok{,])), }\StringTok{"square meters}\SpecialCharTok{\textbackslash{}n}\StringTok{"}\NormalTok{)}
\end{Highlighting}
\end{Shaded}

\begin{verbatim}
  - Cell area: 250000 square meters
\end{verbatim}

\subsubsection{2.2 \textbar{} Aggregate
\textless Burglaries\textgreater{} to
Grid}\label{aggregate-burglaries-to-grid}

\begin{Shaded}
\begin{Highlighting}[]
\CommentTok{\# Spatial join: which cell contains each burglary?}
\NormalTok{burglaries\_fishnet }\OtherTok{\textless{}{-}} \FunctionTok{st\_join}\NormalTok{(burglaries, fishnet, }\AttributeTok{join =}\NormalTok{ st\_within) }\SpecialCharTok{\%\textgreater{}\%}
  \FunctionTok{st\_drop\_geometry}\NormalTok{() }\SpecialCharTok{\%\textgreater{}\%}
  \FunctionTok{group\_by}\NormalTok{(uniqueID) }\SpecialCharTok{\%\textgreater{}\%}
  \FunctionTok{summarize}\NormalTok{(}\AttributeTok{countBurglaries =} \FunctionTok{n}\NormalTok{())}

\CommentTok{\# Join back to fishnet (cells with 0 burglaries will be NA)}
\NormalTok{fishnet }\OtherTok{\textless{}{-}}\NormalTok{ fishnet }\SpecialCharTok{\%\textgreater{}\%}
  \FunctionTok{left\_join}\NormalTok{(burglaries\_fishnet, }\AttributeTok{by =} \StringTok{"uniqueID"}\NormalTok{) }\SpecialCharTok{\%\textgreater{}\%}
  \FunctionTok{mutate}\NormalTok{(}\AttributeTok{countBurglaries =} \FunctionTok{replace\_na}\NormalTok{(countBurglaries, }\DecValTok{0}\NormalTok{))}

\CommentTok{\# Summary statistics}
\FunctionTok{cat}\NormalTok{(}\StringTok{"}\SpecialCharTok{\textbackslash{}n}\StringTok{Burglary count distribution:}\SpecialCharTok{\textbackslash{}n}\StringTok{"}\NormalTok{)}
\end{Highlighting}
\end{Shaded}

\begin{verbatim}

Burglary count distribution:
\end{verbatim}

\begin{Shaded}
\begin{Highlighting}[]
\FunctionTok{summary}\NormalTok{(fishnet}\SpecialCharTok{$}\NormalTok{countBurglaries)}
\end{Highlighting}
\end{Shaded}

\begin{verbatim}
   Min. 1st Qu.  Median    Mean 3rd Qu.    Max. 
  0.000   0.000   2.000   3.042   5.000  40.000 
\end{verbatim}

\begin{Shaded}
\begin{Highlighting}[]
\FunctionTok{cat}\NormalTok{(}\StringTok{"}\SpecialCharTok{\textbackslash{}n}\StringTok{Cells with zero burglaries:"}\NormalTok{, }
    \FunctionTok{sum}\NormalTok{(fishnet}\SpecialCharTok{$}\NormalTok{countBurglaries }\SpecialCharTok{==} \DecValTok{0}\NormalTok{), }
    \StringTok{"/"}\NormalTok{, }\FunctionTok{nrow}\NormalTok{(fishnet),}
    \StringTok{"("}\NormalTok{, }\FunctionTok{round}\NormalTok{(}\DecValTok{100} \SpecialCharTok{*} \FunctionTok{sum}\NormalTok{(fishnet}\SpecialCharTok{$}\NormalTok{countBurglaries }\SpecialCharTok{==} \DecValTok{0}\NormalTok{) }\SpecialCharTok{/} \FunctionTok{nrow}\NormalTok{(fishnet), }\DecValTok{1}\NormalTok{), }\StringTok{"\%)}\SpecialCharTok{\textbackslash{}n}\StringTok{"}\NormalTok{)}
\end{Highlighting}
\end{Shaded}

\begin{verbatim}

Cells with zero burglaries: 781 / 2458 ( 31.8 %)
\end{verbatim}

\subsubsection{2.3 \textbar{} Visualize Count
Distribution}\label{visualize-count-distribution}

\begin{Shaded}
\begin{Highlighting}[]
\CommentTok{\# Visualize aggregated counts}
\FunctionTok{ggplot}\NormalTok{() }\SpecialCharTok{+}
  \FunctionTok{geom\_sf}\NormalTok{(}\AttributeTok{data =}\NormalTok{ fishnet, }\FunctionTok{aes}\NormalTok{(}\AttributeTok{fill =}\NormalTok{ countBurglaries), }\AttributeTok{color =} \ConstantTok{NA}\NormalTok{) }\SpecialCharTok{+}
  \FunctionTok{geom\_sf}\NormalTok{(}\AttributeTok{data =}\NormalTok{ chicagoBoundary, }\AttributeTok{fill =} \ConstantTok{NA}\NormalTok{, }\AttributeTok{color =} \StringTok{"white"}\NormalTok{, }\AttributeTok{linewidth =} \DecValTok{1}\NormalTok{) }\SpecialCharTok{+}
  \FunctionTok{scale\_fill\_viridis\_c}\NormalTok{(}
    \AttributeTok{name =} \StringTok{"Burglaries"}\NormalTok{,}
    \AttributeTok{option =} \StringTok{"plasma"}\NormalTok{,}
    \AttributeTok{trans =} \StringTok{"sqrt"}\NormalTok{,  }\CommentTok{\# Square root for better visualization of skewed data}
    \AttributeTok{breaks =} \FunctionTok{c}\NormalTok{(}\DecValTok{0}\NormalTok{, }\DecValTok{1}\NormalTok{, }\DecValTok{5}\NormalTok{, }\DecValTok{10}\NormalTok{, }\DecValTok{20}\NormalTok{, }\DecValTok{40}\NormalTok{)}
\NormalTok{  ) }\SpecialCharTok{+}
  \FunctionTok{labs}\NormalTok{(}
    \AttributeTok{title =} \StringTok{"Burglary Counts by Grid Cell"}\NormalTok{,}
    \AttributeTok{subtitle =} \StringTok{"500m x 500m cells, Chicago 2017"}
\NormalTok{  ) }\SpecialCharTok{+}
  \FunctionTok{theme\_crime}\NormalTok{()}
\end{Highlighting}
\end{Shaded}

\pandocbounded{\includegraphics[keepaspectratio]{assignment4_template_files/figure-pdf/visualize-fishnet-1.pdf}}

\section{Part 3: Create a Kernel Density Estimate (KDE)
Baseline}\label{part-3-create-a-kernel-density-estimate-kde-baseline}

\begin{itemize}
\tightlist
\item
  KDE represents our null hypothesis: burglaries happen where they
  happened before, with no other information
\end{itemize}

\begin{Shaded}
\begin{Highlighting}[]
\CommentTok{\# Convert burglaries to ppp (point pattern) format for spatstat}
\NormalTok{burglaries\_ppp }\OtherTok{\textless{}{-}} \FunctionTok{as.ppp}\NormalTok{(}
  \FunctionTok{st\_coordinates}\NormalTok{(burglaries),}
  \AttributeTok{W =} \FunctionTok{as.owin}\NormalTok{(}\FunctionTok{st\_bbox}\NormalTok{(chicagoBoundary))}
\NormalTok{)}

\CommentTok{\# Calculate KDE with 1km bandwidth}
\NormalTok{kde\_burglaries }\OtherTok{\textless{}{-}} \FunctionTok{density.ppp}\NormalTok{(}
\NormalTok{  burglaries\_ppp,}
  \AttributeTok{sigma =} \DecValTok{1000}\NormalTok{,  }\CommentTok{\# 1km bandwidth}
  \AttributeTok{edge =} \ConstantTok{TRUE}    \CommentTok{\# Edge correction}
\NormalTok{)}

\CommentTok{\# Convert to terra raster (modern approach, not raster::raster)}
\NormalTok{kde\_raster }\OtherTok{\textless{}{-}} \FunctionTok{rast}\NormalTok{(kde\_burglaries)}

\CommentTok{\# Extract KDE values to fishnet cells}
\NormalTok{fishnet }\OtherTok{\textless{}{-}}\NormalTok{ fishnet }\SpecialCharTok{\%\textgreater{}\%}
  \FunctionTok{mutate}\NormalTok{(}
    \AttributeTok{kde\_value =}\NormalTok{ terra}\SpecialCharTok{::}\FunctionTok{extract}\NormalTok{(}
\NormalTok{      kde\_raster,}
      \FunctionTok{vect}\NormalTok{(fishnet),}
      \AttributeTok{fun =}\NormalTok{ mean,}
      \AttributeTok{na.rm =} \ConstantTok{TRUE}
\NormalTok{    )[, }\DecValTok{2}\NormalTok{]  }\CommentTok{\# Extract just the values column}
\NormalTok{  )}

\FunctionTok{cat}\NormalTok{(}\StringTok{"✓ Calculated KDE baseline}\SpecialCharTok{\textbackslash{}n}\StringTok{"}\NormalTok{)}
\end{Highlighting}
\end{Shaded}

\begin{verbatim}
✓ Calculated KDE baseline
\end{verbatim}

\begin{Shaded}
\begin{Highlighting}[]
\FunctionTok{ggplot}\NormalTok{() }\SpecialCharTok{+}
  \FunctionTok{geom\_sf}\NormalTok{(}\AttributeTok{data =}\NormalTok{ fishnet, }\FunctionTok{aes}\NormalTok{(}\AttributeTok{fill =}\NormalTok{ kde\_value), }\AttributeTok{color =} \ConstantTok{NA}\NormalTok{) }\SpecialCharTok{+}
  \FunctionTok{geom\_sf}\NormalTok{(}\AttributeTok{data =}\NormalTok{ chicagoBoundary, }\AttributeTok{fill =} \ConstantTok{NA}\NormalTok{, }\AttributeTok{color =} \StringTok{"white"}\NormalTok{, }\AttributeTok{linewidth =} \DecValTok{1}\NormalTok{) }\SpecialCharTok{+}
  \FunctionTok{scale\_fill\_viridis\_c}\NormalTok{(}
    \AttributeTok{name =} \StringTok{"KDE Value"}\NormalTok{,}
    \AttributeTok{option =} \StringTok{"plasma"}
\NormalTok{  ) }\SpecialCharTok{+}
  \FunctionTok{labs}\NormalTok{(}
    \AttributeTok{title =} \StringTok{"Kernel Density Estimation Baseline"}\NormalTok{,}
    \AttributeTok{subtitle =} \StringTok{"Simple spatial smoothing of burglary locations"}
\NormalTok{  ) }\SpecialCharTok{+}
  \FunctionTok{theme\_crime}\NormalTok{()}
\end{Highlighting}
\end{Shaded}

\pandocbounded{\includegraphics[keepaspectratio]{assignment4_template_files/figure-pdf/visualize-kde-1.pdf}}

\section{Part 4: Create Spatial Predictor
Variables}\label{part-4-create-spatial-predictor-variables}

Now we'll create features that might help predict burglaries. We'll use
``broken windows theory'' logic: signs of disorder predict crime.

\subsubsection{4.1 \textbar{} Load 311 Abandoned Vehicle
Calls}\label{load-311-abandoned-vehicle-calls}

\begin{Shaded}
\begin{Highlighting}[]
\NormalTok{abandoned\_cars }\OtherTok{\textless{}{-}} \FunctionTok{read\_csv}\NormalTok{(}\StringTok{"data/abandoned\_cars\_2017.csv"}\NormalTok{)}\SpecialCharTok{\%\textgreater{}\%}
  \FunctionTok{filter}\NormalTok{(}\SpecialCharTok{!}\FunctionTok{is.na}\NormalTok{(Latitude), }\SpecialCharTok{!}\FunctionTok{is.na}\NormalTok{(Longitude)) }\SpecialCharTok{\%\textgreater{}\%}
  \FunctionTok{st\_as\_sf}\NormalTok{(}\AttributeTok{coords =} \FunctionTok{c}\NormalTok{(}\StringTok{"Longitude"}\NormalTok{, }\StringTok{"Latitude"}\NormalTok{), }\AttributeTok{crs =} \DecValTok{4326}\NormalTok{) }\SpecialCharTok{\%\textgreater{}\%}
  \FunctionTok{st\_transform}\NormalTok{(}\StringTok{\textquotesingle{}ESRI:102271\textquotesingle{}}\NormalTok{)}

\FunctionTok{cat}\NormalTok{(}\StringTok{"✓ Loaded abandoned vehicle calls}\SpecialCharTok{\textbackslash{}n}\StringTok{"}\NormalTok{)}
\end{Highlighting}
\end{Shaded}

\begin{verbatim}
✓ Loaded abandoned vehicle calls
\end{verbatim}

\begin{Shaded}
\begin{Highlighting}[]
\FunctionTok{cat}\NormalTok{(}\StringTok{"  {-} Number of calls:"}\NormalTok{, }\FunctionTok{nrow}\NormalTok{(abandoned\_cars), }\StringTok{"}\SpecialCharTok{\textbackslash{}n}\StringTok{"}\NormalTok{)}
\end{Highlighting}
\end{Shaded}

\begin{verbatim}
  - Number of calls: 31390 
\end{verbatim}

\begin{tcolorbox}[enhanced jigsaw, colback=white, toprule=.15mm, rightrule=.15mm, bottomrule=.15mm, left=2mm, opacitybacktitle=0.6, coltitle=black, colbacktitle=quarto-callout-note-color!10!white, titlerule=0mm, opacityback=0, bottomtitle=1mm, toptitle=1mm, title=\textcolor{quarto-callout-note-color}{\faInfo}\hspace{0.5em}{Data Loading Note}, arc=.35mm, leftrule=.75mm, breakable, colframe=quarto-callout-note-color-frame]

The data was downloaded from Chicago's Open Data Portal. You can now
request an api from the Chicago portal and tap into the data there.

\textbf{Consider:} How might the 311 reporting system itself be biased?
Who calls 311? What neighborhoods have better 311 awareness?

\end{tcolorbox}

\subsubsection{4.2 \textbar{} Count of Abandoned Cars per
Cell}\label{count-of-abandoned-cars-per-cell}

\begin{Shaded}
\begin{Highlighting}[]
\CommentTok{\# Aggregate abandoned car calls to fishnet}
\NormalTok{abandoned\_fishnet }\OtherTok{\textless{}{-}} \FunctionTok{st\_join}\NormalTok{(abandoned\_cars, fishnet, }\AttributeTok{join =}\NormalTok{ st\_within) }\SpecialCharTok{\%\textgreater{}\%}
  \FunctionTok{st\_drop\_geometry}\NormalTok{() }\SpecialCharTok{\%\textgreater{}\%}
  \FunctionTok{group\_by}\NormalTok{(uniqueID) }\SpecialCharTok{\%\textgreater{}\%}
  \FunctionTok{summarize}\NormalTok{(}\AttributeTok{abandoned\_cars =} \FunctionTok{n}\NormalTok{())}

\CommentTok{\# Join to fishnet}
\NormalTok{fishnet }\OtherTok{\textless{}{-}}\NormalTok{ fishnet }\SpecialCharTok{\%\textgreater{}\%}
  \FunctionTok{left\_join}\NormalTok{(abandoned\_fishnet, }\AttributeTok{by =} \StringTok{"uniqueID"}\NormalTok{) }\SpecialCharTok{\%\textgreater{}\%}
  \FunctionTok{mutate}\NormalTok{(}\AttributeTok{abandoned\_cars =} \FunctionTok{replace\_na}\NormalTok{(abandoned\_cars, }\DecValTok{0}\NormalTok{))}

\FunctionTok{cat}\NormalTok{(}\StringTok{"Abandoned car distribution:}\SpecialCharTok{\textbackslash{}n}\StringTok{"}\NormalTok{)}
\end{Highlighting}
\end{Shaded}

\begin{verbatim}
Abandoned car distribution:
\end{verbatim}

\begin{Shaded}
\begin{Highlighting}[]
\FunctionTok{summary}\NormalTok{(fishnet}\SpecialCharTok{$}\NormalTok{abandoned\_cars)}
\end{Highlighting}
\end{Shaded}

\begin{verbatim}
   Min. 1st Qu.  Median    Mean 3rd Qu.    Max. 
   0.00    2.00    9.00   12.74   19.00  123.00 
\end{verbatim}

\begin{Shaded}
\begin{Highlighting}[]
\NormalTok{p1 }\OtherTok{\textless{}{-}} \FunctionTok{ggplot}\NormalTok{() }\SpecialCharTok{+}
  \FunctionTok{geom\_sf}\NormalTok{(}\AttributeTok{data =}\NormalTok{ fishnet, }\FunctionTok{aes}\NormalTok{(}\AttributeTok{fill =}\NormalTok{ abandoned\_cars), }\AttributeTok{color =} \ConstantTok{NA}\NormalTok{) }\SpecialCharTok{+}
  \FunctionTok{scale\_fill\_viridis\_c}\NormalTok{(}\AttributeTok{name =} \StringTok{"Count"}\NormalTok{, }\AttributeTok{option =} \StringTok{"magma"}\NormalTok{) }\SpecialCharTok{+}
  \FunctionTok{labs}\NormalTok{(}\AttributeTok{title =} \StringTok{"Abandoned Vehicle 311 Calls"}\NormalTok{) }\SpecialCharTok{+}
  \FunctionTok{theme\_crime}\NormalTok{()}

\NormalTok{p2 }\OtherTok{\textless{}{-}} \FunctionTok{ggplot}\NormalTok{() }\SpecialCharTok{+}
  \FunctionTok{geom\_sf}\NormalTok{(}\AttributeTok{data =}\NormalTok{ fishnet, }\FunctionTok{aes}\NormalTok{(}\AttributeTok{fill =}\NormalTok{ countBurglaries), }\AttributeTok{color =} \ConstantTok{NA}\NormalTok{) }\SpecialCharTok{+}
  \FunctionTok{scale\_fill\_viridis\_c}\NormalTok{(}\AttributeTok{name =} \StringTok{"Count"}\NormalTok{, }\AttributeTok{option =} \StringTok{"plasma"}\NormalTok{) }\SpecialCharTok{+}
  \FunctionTok{labs}\NormalTok{(}\AttributeTok{title =} \StringTok{"Burglaries"}\NormalTok{) }\SpecialCharTok{+}
  \FunctionTok{theme\_crime}\NormalTok{()}

\NormalTok{p1 }\SpecialCharTok{+}\NormalTok{ p2 }\SpecialCharTok{+}
  \FunctionTok{plot\_annotation}\NormalTok{(}\AttributeTok{title =} \StringTok{"Are abandoned cars and burglaries correlated?"}\NormalTok{)}
\end{Highlighting}
\end{Shaded}

\pandocbounded{\includegraphics[keepaspectratio]{assignment4_template_files/figure-pdf/visualize-abandoned-cars-1.pdf}}

\subsubsection{4.3 \textbar{} k-Nearest Neighbors (kNN)
Features}\label{k-nearest-neighbors-knn-features}

Count in a cell is one measure. Distance to the nearest 3 abandoned cars
captures local context.

\begin{Shaded}
\begin{Highlighting}[]
\CommentTok{\# Calculate mean distance to 3 nearest abandoned cars}
\CommentTok{\# (Do this OUTSIDE of mutate to avoid sf conflicts)}

\CommentTok{\# Get coordinates}
\NormalTok{fishnet\_coords }\OtherTok{\textless{}{-}} \FunctionTok{st\_coordinates}\NormalTok{(}\FunctionTok{st\_centroid}\NormalTok{(fishnet))}
\NormalTok{abandoned\_coords }\OtherTok{\textless{}{-}} \FunctionTok{st\_coordinates}\NormalTok{(abandoned\_cars)}

\CommentTok{\# Calculate k nearest neighbors and distances}
\NormalTok{nn\_result }\OtherTok{\textless{}{-}} \FunctionTok{get.knnx}\NormalTok{(abandoned\_coords, fishnet\_coords, }\AttributeTok{k =} \DecValTok{3}\NormalTok{)}

\CommentTok{\# Add to fishnet}
\NormalTok{fishnet }\OtherTok{\textless{}{-}}\NormalTok{ fishnet }\SpecialCharTok{\%\textgreater{}\%}
  \FunctionTok{mutate}\NormalTok{(}
    \AttributeTok{abandoned\_cars.nn =} \FunctionTok{rowMeans}\NormalTok{(nn\_result}\SpecialCharTok{$}\NormalTok{nn.dist)}
\NormalTok{  )}

\FunctionTok{cat}\NormalTok{(}\StringTok{"✓ Calculated nearest neighbor distances}\SpecialCharTok{\textbackslash{}n}\StringTok{"}\NormalTok{)}
\end{Highlighting}
\end{Shaded}

\begin{verbatim}
✓ Calculated nearest neighbor distances
\end{verbatim}

\begin{Shaded}
\begin{Highlighting}[]
\FunctionTok{summary}\NormalTok{(fishnet}\SpecialCharTok{$}\NormalTok{abandoned\_cars.nn)}
\end{Highlighting}
\end{Shaded}

\begin{verbatim}
    Min.  1st Qu.   Median     Mean  3rd Qu.     Max. 
   4.386   88.247  143.293  246.946  271.283 2195.753 
\end{verbatim}

\subsubsection{4.4 \textbar{} Local Moran's I; Identify Hotspots;
Distance to Hot
Spots}\label{local-morans-i-identify-hotspots-distance-to-hot-spots}

Let's identify clusters of abandoned cars using Local Moran's I, then
calculate distance to these hot spots.

\begin{Shaded}
\begin{Highlighting}[]
\CommentTok{\# Function to calculate Local Moran\textquotesingle{}s I}
\NormalTok{calculate\_local\_morans }\OtherTok{\textless{}{-}} \ControlFlowTok{function}\NormalTok{(data, variable, }\AttributeTok{k =} \DecValTok{5}\NormalTok{) \{}
  
  \CommentTok{\# Create spatial weights}
\NormalTok{  coords }\OtherTok{\textless{}{-}} \FunctionTok{st\_coordinates}\NormalTok{(}\FunctionTok{st\_centroid}\NormalTok{(data))}
\NormalTok{  neighbors }\OtherTok{\textless{}{-}} \FunctionTok{knn2nb}\NormalTok{(}\FunctionTok{knearneigh}\NormalTok{(coords, }\AttributeTok{k =}\NormalTok{ k))}
\NormalTok{  weights }\OtherTok{\textless{}{-}} \FunctionTok{nb2listw}\NormalTok{(neighbors, }\AttributeTok{style =} \StringTok{"W"}\NormalTok{, }\AttributeTok{zero.policy =} \ConstantTok{TRUE}\NormalTok{)}
  
  \CommentTok{\# Calculate Local Moran\textquotesingle{}s I}
\NormalTok{  local\_moran }\OtherTok{\textless{}{-}} \FunctionTok{localmoran}\NormalTok{(data[[variable]], weights)}
  
  \CommentTok{\# Classify clusters}
\NormalTok{  mean\_val }\OtherTok{\textless{}{-}} \FunctionTok{mean}\NormalTok{(data[[variable]], }\AttributeTok{na.rm =} \ConstantTok{TRUE}\NormalTok{)}
  
\NormalTok{  data }\SpecialCharTok{\%\textgreater{}\%}
    \FunctionTok{mutate}\NormalTok{(}
      \AttributeTok{local\_i =}\NormalTok{ local\_moran[, }\DecValTok{1}\NormalTok{],}
      \AttributeTok{p\_value =}\NormalTok{ local\_moran[, }\DecValTok{5}\NormalTok{],}
      \AttributeTok{is\_significant =}\NormalTok{ p\_value }\SpecialCharTok{\textless{}} \FloatTok{0.05}\NormalTok{,}
      
      \AttributeTok{moran\_class =} \FunctionTok{case\_when}\NormalTok{(}
        \SpecialCharTok{!}\NormalTok{is\_significant }\SpecialCharTok{\textasciitilde{}} \StringTok{"Not Significant"}\NormalTok{,}
\NormalTok{        local\_i }\SpecialCharTok{\textgreater{}} \DecValTok{0} \SpecialCharTok{\&}\NormalTok{ .data[[variable]] }\SpecialCharTok{\textgreater{}}\NormalTok{ mean\_val }\SpecialCharTok{\textasciitilde{}} \StringTok{"High{-}High"}\NormalTok{,}
\NormalTok{        local\_i }\SpecialCharTok{\textgreater{}} \DecValTok{0} \SpecialCharTok{\&}\NormalTok{ .data[[variable]] }\SpecialCharTok{\textless{}=}\NormalTok{ mean\_val }\SpecialCharTok{\textasciitilde{}} \StringTok{"Low{-}Low"}\NormalTok{,}
\NormalTok{        local\_i }\SpecialCharTok{\textless{}} \DecValTok{0} \SpecialCharTok{\&}\NormalTok{ .data[[variable]] }\SpecialCharTok{\textgreater{}}\NormalTok{ mean\_val }\SpecialCharTok{\textasciitilde{}} \StringTok{"High{-}Low"}\NormalTok{,}
\NormalTok{        local\_i }\SpecialCharTok{\textless{}} \DecValTok{0} \SpecialCharTok{\&}\NormalTok{ .data[[variable]] }\SpecialCharTok{\textless{}=}\NormalTok{ mean\_val }\SpecialCharTok{\textasciitilde{}} \StringTok{"Low{-}High"}\NormalTok{,}
        \ConstantTok{TRUE} \SpecialCharTok{\textasciitilde{}} \StringTok{"Not Significant"}
\NormalTok{      )}
\NormalTok{    )}
\NormalTok{\}}

\CommentTok{\# Apply to abandoned cars}
\NormalTok{fishnet }\OtherTok{\textless{}{-}} \FunctionTok{calculate\_local\_morans}\NormalTok{(fishnet, }\StringTok{"abandoned\_cars"}\NormalTok{, }\AttributeTok{k =} \DecValTok{5}\NormalTok{)}
\end{Highlighting}
\end{Shaded}

\begin{Shaded}
\begin{Highlighting}[]
\CommentTok{\# Visualize hot spots}
\FunctionTok{ggplot}\NormalTok{() }\SpecialCharTok{+}
  \FunctionTok{geom\_sf}\NormalTok{(}
    \AttributeTok{data =}\NormalTok{ fishnet, }
    \FunctionTok{aes}\NormalTok{(}\AttributeTok{fill =}\NormalTok{ moran\_class), }
    \AttributeTok{color =} \ConstantTok{NA}
\NormalTok{  ) }\SpecialCharTok{+}
  \FunctionTok{scale\_fill\_manual}\NormalTok{(}
    \AttributeTok{values =} \FunctionTok{c}\NormalTok{(}
      \StringTok{"High{-}High"} \OtherTok{=} \StringTok{"\#d7191c"}\NormalTok{,}
      \StringTok{"High{-}Low"} \OtherTok{=} \StringTok{"\#fdae61"}\NormalTok{,}
      \StringTok{"Low{-}High"} \OtherTok{=} \StringTok{"\#abd9e9"}\NormalTok{,}
      \StringTok{"Low{-}Low"} \OtherTok{=} \StringTok{"\#2c7bb6"}\NormalTok{,}
      \StringTok{"Not Significant"} \OtherTok{=} \StringTok{"gray90"}
\NormalTok{    ),}
    \AttributeTok{name =} \StringTok{"Cluster Type"}
\NormalTok{  ) }\SpecialCharTok{+}
  \FunctionTok{labs}\NormalTok{(}
    \AttributeTok{title =} \StringTok{"Local Moran\textquotesingle{}s I: Abandoned Car Clusters"}\NormalTok{,}
    \AttributeTok{subtitle =} \StringTok{"High{-}High = Hot spots of disorder"}
\NormalTok{  ) }\SpecialCharTok{+}
  \FunctionTok{theme\_crime}\NormalTok{()}
\end{Highlighting}
\end{Shaded}

\pandocbounded{\includegraphics[keepaspectratio]{assignment4_template_files/figure-pdf/visualize-morans-1.pdf}}

\begin{Shaded}
\begin{Highlighting}[]
\CommentTok{\# Get centroids of "High{-}High" cells (hot spots)}
\NormalTok{hotspots }\OtherTok{\textless{}{-}}\NormalTok{ fishnet }\SpecialCharTok{\%\textgreater{}\%}
  \FunctionTok{filter}\NormalTok{(moran\_class }\SpecialCharTok{==} \StringTok{"High{-}High"}\NormalTok{) }\SpecialCharTok{\%\textgreater{}\%}
  \FunctionTok{st\_centroid}\NormalTok{()}

\CommentTok{\# Calculate distance from each cell to nearest hot spot}
\ControlFlowTok{if}\NormalTok{ (}\FunctionTok{nrow}\NormalTok{(hotspots) }\SpecialCharTok{\textgreater{}} \DecValTok{0}\NormalTok{) \{}
\NormalTok{  fishnet }\OtherTok{\textless{}{-}}\NormalTok{ fishnet }\SpecialCharTok{\%\textgreater{}\%}
    \FunctionTok{mutate}\NormalTok{(}
      \AttributeTok{dist\_to\_hotspot =} \FunctionTok{as.numeric}\NormalTok{(}
        \FunctionTok{st\_distance}\NormalTok{(}\FunctionTok{st\_centroid}\NormalTok{(fishnet), hotspots }\SpecialCharTok{\%\textgreater{}\%} \FunctionTok{st\_union}\NormalTok{())}
\NormalTok{      )}
\NormalTok{    )}
  
  \FunctionTok{cat}\NormalTok{(}\StringTok{"✓ Calculated distance to abandoned car hot spots}\SpecialCharTok{\textbackslash{}n}\StringTok{"}\NormalTok{)}
  \FunctionTok{cat}\NormalTok{(}\StringTok{"  {-} Number of hot spot cells:"}\NormalTok{, }\FunctionTok{nrow}\NormalTok{(hotspots), }\StringTok{"}\SpecialCharTok{\textbackslash{}n}\StringTok{"}\NormalTok{)}
\NormalTok{\} }\ControlFlowTok{else}\NormalTok{ \{}
\NormalTok{  fishnet }\OtherTok{\textless{}{-}}\NormalTok{ fishnet }\SpecialCharTok{\%\textgreater{}\%}
    \FunctionTok{mutate}\NormalTok{(}\AttributeTok{dist\_to\_hotspot =} \DecValTok{0}\NormalTok{)}
  \FunctionTok{cat}\NormalTok{(}\StringTok{"⚠ No significant hot spots found}\SpecialCharTok{\textbackslash{}n}\StringTok{"}\NormalTok{)}
\NormalTok{\}}
\end{Highlighting}
\end{Shaded}

\begin{verbatim}
✓ Calculated distance to abandoned car hot spots
  - Number of hot spot cells: 275 
\end{verbatim}

\begin{center}\rule{0.5\linewidth}{0.5pt}\end{center}

\section{Part 5: Join Police Districts for
Cross-Validation}\label{part-5-join-police-districts-for-cross-validation}

We'll use police districts for our spatial cross-validation.

\begin{Shaded}
\begin{Highlighting}[]
\CommentTok{\# Join district information to fishnet}
\NormalTok{fishnet }\OtherTok{\textless{}{-}} \FunctionTok{st\_join}\NormalTok{(}
\NormalTok{  fishnet,}
\NormalTok{  policeDistricts,}
  \AttributeTok{join =}\NormalTok{ st\_within,}
  \AttributeTok{left =} \ConstantTok{TRUE}
\NormalTok{) }\SpecialCharTok{\%\textgreater{}\%}
  \FunctionTok{filter}\NormalTok{(}\SpecialCharTok{!}\FunctionTok{is.na}\NormalTok{(District))  }\CommentTok{\# Remove cells outside districts}

\FunctionTok{cat}\NormalTok{(}\StringTok{"✓ Joined police districts}\SpecialCharTok{\textbackslash{}n}\StringTok{"}\NormalTok{)}
\end{Highlighting}
\end{Shaded}

\begin{verbatim}
✓ Joined police districts
\end{verbatim}

\begin{Shaded}
\begin{Highlighting}[]
\FunctionTok{cat}\NormalTok{(}\StringTok{"  {-} Districts:"}\NormalTok{, }\FunctionTok{length}\NormalTok{(}\FunctionTok{unique}\NormalTok{(fishnet}\SpecialCharTok{$}\NormalTok{District)), }\StringTok{"}\SpecialCharTok{\textbackslash{}n}\StringTok{"}\NormalTok{)}
\end{Highlighting}
\end{Shaded}

\begin{verbatim}
  - Districts: 22 
\end{verbatim}

\begin{Shaded}
\begin{Highlighting}[]
\FunctionTok{cat}\NormalTok{(}\StringTok{"  {-} Cells:"}\NormalTok{, }\FunctionTok{nrow}\NormalTok{(fishnet), }\StringTok{"}\SpecialCharTok{\textbackslash{}n}\StringTok{"}\NormalTok{)}
\end{Highlighting}
\end{Shaded}

\begin{verbatim}
  - Cells: 1708 
\end{verbatim}

\section{Part 6: Count Regression Model
Fitting}\label{part-6-count-regression-model-fitting}

\subsubsection{6.1 \textbar{} Poisson
Regression}\label{poisson-regression}

Burglary counts are count data (0, 1, 2, 3\ldots). We'll use
\textbf{Poisson regression}.

\begin{Shaded}
\begin{Highlighting}[]
\CommentTok{\# Create clean modeling dataset}
\NormalTok{fishnet\_model }\OtherTok{\textless{}{-}}\NormalTok{ fishnet }\SpecialCharTok{\%\textgreater{}\%}
  \FunctionTok{st\_drop\_geometry}\NormalTok{() }\SpecialCharTok{\%\textgreater{}\%}
\NormalTok{  dplyr}\SpecialCharTok{::}\FunctionTok{select}\NormalTok{(}
\NormalTok{    uniqueID,}
\NormalTok{    District,}
\NormalTok{    countBurglaries,}
\NormalTok{    abandoned\_cars,}
\NormalTok{    abandoned\_cars.nn,}
\NormalTok{    dist\_to\_hotspot}
\NormalTok{  ) }\SpecialCharTok{\%\textgreater{}\%}
  \FunctionTok{na.omit}\NormalTok{()  }\CommentTok{\# Remove any remaining NAs}

\FunctionTok{cat}\NormalTok{(}\StringTok{"✓ Prepared modeling data}\SpecialCharTok{\textbackslash{}n}\StringTok{"}\NormalTok{)}
\end{Highlighting}
\end{Shaded}

\begin{verbatim}
✓ Prepared modeling data
\end{verbatim}

\begin{Shaded}
\begin{Highlighting}[]
\FunctionTok{cat}\NormalTok{(}\StringTok{"  {-} Observations:"}\NormalTok{, }\FunctionTok{nrow}\NormalTok{(fishnet\_model), }\StringTok{"}\SpecialCharTok{\textbackslash{}n}\StringTok{"}\NormalTok{)}
\end{Highlighting}
\end{Shaded}

\begin{verbatim}
  - Observations: 1708 
\end{verbatim}

\begin{Shaded}
\begin{Highlighting}[]
\FunctionTok{cat}\NormalTok{(}\StringTok{"  {-} Variables:"}\NormalTok{, }\FunctionTok{ncol}\NormalTok{(fishnet\_model), }\StringTok{"}\SpecialCharTok{\textbackslash{}n}\StringTok{"}\NormalTok{)}
\end{Highlighting}
\end{Shaded}

\begin{verbatim}
  - Variables: 6 
\end{verbatim}

\begin{Shaded}
\begin{Highlighting}[]
\CommentTok{\# Fit Poisson regression}
\NormalTok{model\_poisson }\OtherTok{\textless{}{-}} \FunctionTok{glm}\NormalTok{(}
\NormalTok{  countBurglaries }\SpecialCharTok{\textasciitilde{}}\NormalTok{ abandoned\_cars }\SpecialCharTok{+}\NormalTok{ abandoned\_cars.nn }\SpecialCharTok{+} 
\NormalTok{    dist\_to\_hotspot,}
  \AttributeTok{data =}\NormalTok{ fishnet\_model,}
  \AttributeTok{family =} \StringTok{"poisson"}
\NormalTok{)}

\CommentTok{\# Summary}
\FunctionTok{summary}\NormalTok{(model\_poisson)}
\end{Highlighting}
\end{Shaded}

\begin{verbatim}

Call:
glm(formula = countBurglaries ~ abandoned_cars + abandoned_cars.nn + 
    dist_to_hotspot, family = "poisson", data = fishnet_model)

Coefficients:
                      Estimate   Std. Error z value            Pr(>|z|)    
(Intercept)        1.976262369  0.042512701  46.486 <0.0000000000000002 ***
abandoned_cars    -0.001360741  0.001089805  -1.249               0.212    
abandoned_cars.nn -0.004965200  0.000198914 -24.962 <0.0000000000000002 ***
dist_to_hotspot    0.000002874  0.000006206   0.463               0.643    
---
Signif. codes:  0 '***' 0.001 '**' 0.01 '*' 0.05 '.' 0.1 ' ' 1

(Dispersion parameter for poisson family taken to be 1)

    Null deviance: 6710.3  on 1707  degrees of freedom
Residual deviance: 5070.6  on 1704  degrees of freedom
AIC: 9138.9

Number of Fisher Scoring iterations: 6
\end{verbatim}

\subsubsection{6.2 \textbar{} Check for
Overdispersion}\label{check-for-overdispersion}

Poisson regression assumes mean = variance. Real count data often
violates this (overdispersion).

\begin{Shaded}
\begin{Highlighting}[]
\CommentTok{\# Calculate dispersion parameter}
\NormalTok{dispersion }\OtherTok{\textless{}{-}} \FunctionTok{sum}\NormalTok{(}\FunctionTok{residuals}\NormalTok{(model\_poisson, }\AttributeTok{type =} \StringTok{"pearson"}\NormalTok{)}\SpecialCharTok{\^{}}\DecValTok{2}\NormalTok{) }\SpecialCharTok{/} 
\NormalTok{              model\_poisson}\SpecialCharTok{$}\NormalTok{df.residual}

\FunctionTok{cat}\NormalTok{(}\StringTok{"Dispersion parameter:"}\NormalTok{, }\FunctionTok{round}\NormalTok{(dispersion, }\DecValTok{2}\NormalTok{), }\StringTok{"}\SpecialCharTok{\textbackslash{}n}\StringTok{"}\NormalTok{)}
\end{Highlighting}
\end{Shaded}

\begin{verbatim}
Dispersion parameter: 3.38 
\end{verbatim}

\begin{Shaded}
\begin{Highlighting}[]
\FunctionTok{cat}\NormalTok{(}\StringTok{"Rule of thumb: \textgreater{}1.5 suggests overdispersion}\SpecialCharTok{\textbackslash{}n}\StringTok{"}\NormalTok{)}
\end{Highlighting}
\end{Shaded}

\begin{verbatim}
Rule of thumb: >1.5 suggests overdispersion
\end{verbatim}

\begin{Shaded}
\begin{Highlighting}[]
\ControlFlowTok{if}\NormalTok{ (dispersion }\SpecialCharTok{\textgreater{}} \FloatTok{1.5}\NormalTok{) \{}
  \FunctionTok{cat}\NormalTok{(}\StringTok{"⚠ Overdispersion detected! Consider Negative Binomial model.}\SpecialCharTok{\textbackslash{}n}\StringTok{"}\NormalTok{)}
\NormalTok{\} }\ControlFlowTok{else}\NormalTok{ \{}
  \FunctionTok{cat}\NormalTok{(}\StringTok{"✓ Dispersion looks okay for Poisson model.}\SpecialCharTok{\textbackslash{}n}\StringTok{"}\NormalTok{)}
\NormalTok{\}}
\end{Highlighting}
\end{Shaded}

\begin{verbatim}
⚠ Overdispersion detected! Consider Negative Binomial model.
\end{verbatim}

\subsubsection{6.3 \textbar{} Negative Binomial
Regression}\label{negative-binomial-regression}

If overdispersed, use \textbf{Negative Binomial regression} (more
flexible).

\begin{Shaded}
\begin{Highlighting}[]
\CommentTok{\# Fit Negative Binomial model}
\NormalTok{model\_nb }\OtherTok{\textless{}{-}} \FunctionTok{glm.nb}\NormalTok{(}
\NormalTok{  countBurglaries }\SpecialCharTok{\textasciitilde{}}\NormalTok{ abandoned\_cars }\SpecialCharTok{+}\NormalTok{ abandoned\_cars.nn }\SpecialCharTok{+} 
\NormalTok{    dist\_to\_hotspot,}
  \AttributeTok{data =}\NormalTok{ fishnet\_model}
\NormalTok{)}

\CommentTok{\# Summary}
\FunctionTok{summary}\NormalTok{(model\_nb)}
\end{Highlighting}
\end{Shaded}

\begin{verbatim}

Call:
glm.nb(formula = countBurglaries ~ abandoned_cars + abandoned_cars.nn + 
    dist_to_hotspot, data = fishnet_model, init.theta = 1.603099596, 
    link = log)

Coefficients:
                      Estimate   Std. Error z value            Pr(>|z|)    
(Intercept)        2.092907737  0.077469423  27.016 <0.0000000000000002 ***
abandoned_cars    -0.002006352  0.002091851  -0.959               0.337    
abandoned_cars.nn -0.005844829  0.000321389 -18.186 <0.0000000000000002 ***
dist_to_hotspot    0.000006861  0.000011049   0.621               0.535    
---
Signif. codes:  0 '***' 0.001 '**' 0.01 '*' 0.05 '.' 0.1 ' ' 1

(Dispersion parameter for Negative Binomial(1.6031) family taken to be 1)

    Null deviance: 2534.0  on 1707  degrees of freedom
Residual deviance: 1796.6  on 1704  degrees of freedom
AIC: 7522.6

Number of Fisher Scoring iterations: 1

              Theta:  1.6031 
          Std. Err.:  0.0888 

 2 x log-likelihood:  -7512.5850 
\end{verbatim}

\begin{Shaded}
\begin{Highlighting}[]
\CommentTok{\# Compare AIC (lower is better)}
\FunctionTok{cat}\NormalTok{(}\StringTok{"}\SpecialCharTok{\textbackslash{}n}\StringTok{Model Comparison:}\SpecialCharTok{\textbackslash{}n}\StringTok{"}\NormalTok{)}
\end{Highlighting}
\end{Shaded}

\begin{verbatim}

Model Comparison:
\end{verbatim}

\begin{Shaded}
\begin{Highlighting}[]
\FunctionTok{cat}\NormalTok{(}\StringTok{"Poisson AIC:"}\NormalTok{, }\FunctionTok{round}\NormalTok{(}\FunctionTok{AIC}\NormalTok{(model\_poisson), }\DecValTok{1}\NormalTok{), }\StringTok{"}\SpecialCharTok{\textbackslash{}n}\StringTok{"}\NormalTok{)}
\end{Highlighting}
\end{Shaded}

\begin{verbatim}
Poisson AIC: 9138.9 
\end{verbatim}

\begin{Shaded}
\begin{Highlighting}[]
\FunctionTok{cat}\NormalTok{(}\StringTok{"Negative Binomial AIC:"}\NormalTok{, }\FunctionTok{round}\NormalTok{(}\FunctionTok{AIC}\NormalTok{(model\_nb), }\DecValTok{1}\NormalTok{), }\StringTok{"}\SpecialCharTok{\textbackslash{}n}\StringTok{"}\NormalTok{)}
\end{Highlighting}
\end{Shaded}

\begin{verbatim}
Negative Binomial AIC: 7522.6 
\end{verbatim}

\subsubsection{6.4 \textbar{} Compare Model Fit
(AIC)}\label{compare-model-fit-aic}

\begin{itemize}
\tightlist
\item
  To clarify what does AIC mean??
\end{itemize}

\section{Part 7: Spatial LOGO-CV
(2017)}\label{part-7-spatial-logo-cv-2017}

\textbf{Leave-One-Group-Out (LOGO) Cross-Validation} trains on all
districts except one, then tests on the held-out district.

\begin{Shaded}
\begin{Highlighting}[]
\CommentTok{\# Get unique districts}
\NormalTok{districts }\OtherTok{\textless{}{-}} \FunctionTok{unique}\NormalTok{(fishnet\_model}\SpecialCharTok{$}\NormalTok{District)}
\NormalTok{cv\_results }\OtherTok{\textless{}{-}} \FunctionTok{tibble}\NormalTok{()}

\FunctionTok{cat}\NormalTok{(}\StringTok{"Running LOGO Cross{-}Validation...}\SpecialCharTok{\textbackslash{}n}\StringTok{"}\NormalTok{)}
\end{Highlighting}
\end{Shaded}

\begin{verbatim}
Running LOGO Cross-Validation...
\end{verbatim}

\begin{Shaded}
\begin{Highlighting}[]
\ControlFlowTok{for}\NormalTok{ (i }\ControlFlowTok{in} \FunctionTok{seq\_along}\NormalTok{(districts)) \{}
  
\NormalTok{  test\_district }\OtherTok{\textless{}{-}}\NormalTok{ districts[i]}
  
  \CommentTok{\# Split data}
\NormalTok{  train\_data }\OtherTok{\textless{}{-}}\NormalTok{ fishnet\_model }\SpecialCharTok{\%\textgreater{}\%} \FunctionTok{filter}\NormalTok{(District }\SpecialCharTok{!=}\NormalTok{ test\_district)}
\NormalTok{  test\_data }\OtherTok{\textless{}{-}}\NormalTok{ fishnet\_model }\SpecialCharTok{\%\textgreater{}\%} \FunctionTok{filter}\NormalTok{(District }\SpecialCharTok{==}\NormalTok{ test\_district)}
  
  \CommentTok{\# Fit model on training data}
\NormalTok{  model\_cv }\OtherTok{\textless{}{-}} \FunctionTok{glm.nb}\NormalTok{(}
\NormalTok{    countBurglaries }\SpecialCharTok{\textasciitilde{}}\NormalTok{ abandoned\_cars }\SpecialCharTok{+}\NormalTok{ abandoned\_cars.nn }\SpecialCharTok{+} 
\NormalTok{      dist\_to\_hotspot,}
    \AttributeTok{data =}\NormalTok{ train\_data}
\NormalTok{  )}
  
  \CommentTok{\# Predict on test data}
\NormalTok{  test\_data }\OtherTok{\textless{}{-}}\NormalTok{ test\_data }\SpecialCharTok{\%\textgreater{}\%}
    \FunctionTok{mutate}\NormalTok{(}
      \AttributeTok{prediction =} \FunctionTok{predict}\NormalTok{(model\_cv, test\_data, }\AttributeTok{type =} \StringTok{"response"}\NormalTok{)}
\NormalTok{    )}
  
  \CommentTok{\# Calculate metrics}
\NormalTok{  mae }\OtherTok{\textless{}{-}} \FunctionTok{mean}\NormalTok{(}\FunctionTok{abs}\NormalTok{(test\_data}\SpecialCharTok{$}\NormalTok{countBurglaries }\SpecialCharTok{{-}}\NormalTok{ test\_data}\SpecialCharTok{$}\NormalTok{prediction))}
\NormalTok{  rmse }\OtherTok{\textless{}{-}} \FunctionTok{sqrt}\NormalTok{(}\FunctionTok{mean}\NormalTok{((test\_data}\SpecialCharTok{$}\NormalTok{countBurglaries }\SpecialCharTok{{-}}\NormalTok{ test\_data}\SpecialCharTok{$}\NormalTok{prediction)}\SpecialCharTok{\^{}}\DecValTok{2}\NormalTok{))}
  
  \CommentTok{\# Store results}
\NormalTok{  cv\_results }\OtherTok{\textless{}{-}} \FunctionTok{bind\_rows}\NormalTok{(}
\NormalTok{    cv\_results,}
    \FunctionTok{tibble}\NormalTok{(}
      \AttributeTok{fold =}\NormalTok{ i,}
      \AttributeTok{test\_district =}\NormalTok{ test\_district,}
      \AttributeTok{n\_test =} \FunctionTok{nrow}\NormalTok{(test\_data),}
      \AttributeTok{mae =}\NormalTok{ mae,}
      \AttributeTok{rmse =}\NormalTok{ rmse}
\NormalTok{    )}
\NormalTok{  )}
  
  \FunctionTok{cat}\NormalTok{(}\StringTok{"  Fold"}\NormalTok{, i, }\StringTok{"/"}\NormalTok{, }\FunctionTok{length}\NormalTok{(districts), }\StringTok{"{-} District"}\NormalTok{, test\_district, }
      \StringTok{"{-} MAE:"}\NormalTok{, }\FunctionTok{round}\NormalTok{(mae, }\DecValTok{2}\NormalTok{), }\StringTok{"}\SpecialCharTok{\textbackslash{}n}\StringTok{"}\NormalTok{)}
\NormalTok{\}}
\end{Highlighting}
\end{Shaded}

\begin{verbatim}
  Fold 1 / 22 - District 5 - MAE: 2.04 
  Fold 2 / 22 - District 4 - MAE: 1.84 
  Fold 3 / 22 - District 22 - MAE: 2.26 
  Fold 4 / 22 - District 6 - MAE: 3.3 
  Fold 5 / 22 - District 8 - MAE: 2.53 
  Fold 6 / 22 - District 7 - MAE: 3.08 
  Fold 7 / 22 - District 3 - MAE: 6.05 
  Fold 8 / 22 - District 2 - MAE: 2.69 
  Fold 9 / 22 - District 9 - MAE: 2.16 
  Fold 10 / 22 - District 10 - MAE: 2.19 
  Fold 11 / 22 - District 1 - MAE: 1.76 
  Fold 12 / 22 - District 12 - MAE: 3.1 
  Fold 13 / 22 - District 15 - MAE: 2.08 
  Fold 14 / 22 - District 11 - MAE: 3.19 
  Fold 15 / 22 - District 18 - MAE: 2.75 
  Fold 16 / 22 - District 25 - MAE: 2.75 
  Fold 17 / 22 - District 14 - MAE: 2.96 
  Fold 18 / 22 - District 19 - MAE: 2.1 
  Fold 19 / 22 - District 16 - MAE: 2.98 
  Fold 20 / 22 - District 17 - MAE: 2.17 
  Fold 21 / 22 - District 20 - MAE: 2.68 
  Fold 22 / 22 - District 24 - MAE: 2.65 
\end{verbatim}

\begin{Shaded}
\begin{Highlighting}[]
\CommentTok{\# Overall results}
\FunctionTok{cat}\NormalTok{(}\StringTok{"}\SpecialCharTok{\textbackslash{}n}\StringTok{✓ Cross{-}Validation Complete}\SpecialCharTok{\textbackslash{}n}\StringTok{"}\NormalTok{)}
\end{Highlighting}
\end{Shaded}

\begin{verbatim}

✓ Cross-Validation Complete
\end{verbatim}

\begin{Shaded}
\begin{Highlighting}[]
\FunctionTok{cat}\NormalTok{(}\StringTok{"Mean MAE:"}\NormalTok{, }\FunctionTok{round}\NormalTok{(}\FunctionTok{mean}\NormalTok{(cv\_results}\SpecialCharTok{$}\NormalTok{mae), }\DecValTok{2}\NormalTok{), }\StringTok{"}\SpecialCharTok{\textbackslash{}n}\StringTok{"}\NormalTok{)}
\end{Highlighting}
\end{Shaded}

\begin{verbatim}
Mean MAE: 2.7 
\end{verbatim}

\begin{Shaded}
\begin{Highlighting}[]
\FunctionTok{cat}\NormalTok{(}\StringTok{"Mean RMSE:"}\NormalTok{, }\FunctionTok{round}\NormalTok{(}\FunctionTok{mean}\NormalTok{(cv\_results}\SpecialCharTok{$}\NormalTok{rmse), }\DecValTok{2}\NormalTok{), }\StringTok{"}\SpecialCharTok{\textbackslash{}n}\StringTok{"}\NormalTok{)}
\end{Highlighting}
\end{Shaded}

\begin{verbatim}
Mean RMSE: 3.61 
\end{verbatim}

\begin{Shaded}
\begin{Highlighting}[]
\CommentTok{\# Show results}
\NormalTok{cv\_results }\SpecialCharTok{\%\textgreater{}\%}
  \FunctionTok{arrange}\NormalTok{(}\FunctionTok{desc}\NormalTok{(mae)) }\SpecialCharTok{\%\textgreater{}\%}
  \FunctionTok{kable}\NormalTok{(}
    \AttributeTok{digits =} \DecValTok{2}\NormalTok{,}
    \AttributeTok{caption =} \StringTok{"LOGO CV Results by District"}
\NormalTok{  ) }\SpecialCharTok{\%\textgreater{}\%}
  \FunctionTok{kable\_styling}\NormalTok{(}\AttributeTok{bootstrap\_options =} \FunctionTok{c}\NormalTok{(}\StringTok{"striped"}\NormalTok{, }\StringTok{"hover"}\NormalTok{))}
\end{Highlighting}
\end{Shaded}

\begin{longtable}[t]{rlrrr}
\caption{\label{tab:cv-results-table}LOGO CV Results by District}\\
\toprule
fold & test\_district & n\_test & mae & rmse\\
\midrule
7 & 3 & 43 & 6.05 & 8.08\\
4 & 6 & 63 & 3.30 & 4.75\\
14 & 11 & 43 & 3.19 & 4.09\\
12 & 12 & 73 & 3.10 & 4.62\\
6 & 7 & 52 & 3.08 & 4.07\\
\addlinespace
19 & 16 & 129 & 2.98 & 3.48\\
17 & 14 & 46 & 2.96 & 4.24\\
16 & 25 & 85 & 2.75 & 3.62\\
15 & 18 & 30 & 2.75 & 4.15\\
8 & 2 & 56 & 2.69 & 3.60\\
\addlinespace
21 & 20 & 30 & 2.68 & 3.11\\
22 & 24 & 41 & 2.65 & 2.98\\
5 & 8 & 197 & 2.53 & 3.48\\
3 & 22 & 112 & 2.26 & 2.83\\
10 & 10 & 63 & 2.19 & 3.09\\
\addlinespace
20 & 17 & 82 & 2.17 & 2.60\\
9 & 9 & 107 & 2.16 & 2.59\\
18 & 19 & 63 & 2.10 & 2.57\\
13 & 15 & 32 & 2.08 & 2.67\\
1 & 5 & 98 & 2.04 & 3.09\\
\addlinespace
2 & 4 & 235 & 1.84 & 3.69\\
11 & 1 & 28 & 1.76 & 2.11\\
\bottomrule
\end{longtable}

\section{Part 8: Model Predictions and
Comparison}\label{part-8-model-predictions-and-comparison}

\subsubsection{8.1 \textbar{} Generate Final
Predictions}\label{generate-final-predictions}

\begin{Shaded}
\begin{Highlighting}[]
\CommentTok{\# Fit final model on all data}
\NormalTok{final\_model }\OtherTok{\textless{}{-}} \FunctionTok{glm.nb}\NormalTok{(}
\NormalTok{  countBurglaries }\SpecialCharTok{\textasciitilde{}}\NormalTok{ abandoned\_cars }\SpecialCharTok{+}\NormalTok{ abandoned\_cars.nn }\SpecialCharTok{+} 
\NormalTok{    dist\_to\_hotspot,}
  \AttributeTok{data =}\NormalTok{ fishnet\_model}
\NormalTok{)}

\CommentTok{\# Add predictions back to fishnet}
\NormalTok{fishnet }\OtherTok{\textless{}{-}}\NormalTok{ fishnet }\SpecialCharTok{\%\textgreater{}\%}
  \FunctionTok{mutate}\NormalTok{(}
    \AttributeTok{prediction\_nb =} \FunctionTok{predict}\NormalTok{(final\_model, fishnet\_model, }\AttributeTok{type =} \StringTok{"response"}\NormalTok{)[}\FunctionTok{match}\NormalTok{(uniqueID, fishnet\_model}\SpecialCharTok{$}\NormalTok{uniqueID)]}
\NormalTok{  )}

\CommentTok{\# Also add KDE predictions (normalize to same scale as counts)}
\NormalTok{kde\_sum }\OtherTok{\textless{}{-}} \FunctionTok{sum}\NormalTok{(fishnet}\SpecialCharTok{$}\NormalTok{kde\_value, }\AttributeTok{na.rm =} \ConstantTok{TRUE}\NormalTok{)}
\NormalTok{count\_sum }\OtherTok{\textless{}{-}} \FunctionTok{sum}\NormalTok{(fishnet}\SpecialCharTok{$}\NormalTok{countBurglaries, }\AttributeTok{na.rm =} \ConstantTok{TRUE}\NormalTok{)}
\NormalTok{fishnet }\OtherTok{\textless{}{-}}\NormalTok{ fishnet }\SpecialCharTok{\%\textgreater{}\%}
  \FunctionTok{mutate}\NormalTok{(}
    \AttributeTok{prediction\_kde =}\NormalTok{ (kde\_value }\SpecialCharTok{/}\NormalTok{ kde\_sum) }\SpecialCharTok{*}\NormalTok{ count\_sum}
\NormalTok{  )}
\end{Highlighting}
\end{Shaded}

\subsubsection{8.2 \textbar{} Compare Model vs.~KDE
Baseline}\label{compare-model-vs.-kde-baseline}

\begin{Shaded}
\begin{Highlighting}[]
\CommentTok{\# Create three maps}
\NormalTok{p1 }\OtherTok{\textless{}{-}} \FunctionTok{ggplot}\NormalTok{() }\SpecialCharTok{+}
  \FunctionTok{geom\_sf}\NormalTok{(}\AttributeTok{data =}\NormalTok{ fishnet, }\FunctionTok{aes}\NormalTok{(}\AttributeTok{fill =}\NormalTok{ countBurglaries), }\AttributeTok{color =} \ConstantTok{NA}\NormalTok{) }\SpecialCharTok{+}
  \FunctionTok{scale\_fill\_viridis\_c}\NormalTok{(}\AttributeTok{name =} \StringTok{"Count"}\NormalTok{, }\AttributeTok{option =} \StringTok{"plasma"}\NormalTok{, }\AttributeTok{limits =} \FunctionTok{c}\NormalTok{(}\DecValTok{0}\NormalTok{, }\DecValTok{15}\NormalTok{)) }\SpecialCharTok{+}
  \FunctionTok{labs}\NormalTok{(}\AttributeTok{title =} \StringTok{"Actual Burglaries"}\NormalTok{) }\SpecialCharTok{+}
  \FunctionTok{theme\_crime}\NormalTok{()}

\NormalTok{p2 }\OtherTok{\textless{}{-}} \FunctionTok{ggplot}\NormalTok{() }\SpecialCharTok{+}
  \FunctionTok{geom\_sf}\NormalTok{(}\AttributeTok{data =}\NormalTok{ fishnet, }\FunctionTok{aes}\NormalTok{(}\AttributeTok{fill =}\NormalTok{ prediction\_nb), }\AttributeTok{color =} \ConstantTok{NA}\NormalTok{) }\SpecialCharTok{+}
  \FunctionTok{scale\_fill\_viridis\_c}\NormalTok{(}\AttributeTok{name =} \StringTok{"Predicted"}\NormalTok{, }\AttributeTok{option =} \StringTok{"plasma"}\NormalTok{, }\AttributeTok{limits =} \FunctionTok{c}\NormalTok{(}\DecValTok{0}\NormalTok{, }\DecValTok{15}\NormalTok{)) }\SpecialCharTok{+}
  \FunctionTok{labs}\NormalTok{(}\AttributeTok{title =} \StringTok{"Model Predictions (Neg. Binomial)"}\NormalTok{) }\SpecialCharTok{+}
  \FunctionTok{theme\_crime}\NormalTok{()}

\NormalTok{p3 }\OtherTok{\textless{}{-}} \FunctionTok{ggplot}\NormalTok{() }\SpecialCharTok{+}
  \FunctionTok{geom\_sf}\NormalTok{(}\AttributeTok{data =}\NormalTok{ fishnet, }\FunctionTok{aes}\NormalTok{(}\AttributeTok{fill =}\NormalTok{ prediction\_kde), }\AttributeTok{color =} \ConstantTok{NA}\NormalTok{) }\SpecialCharTok{+}
  \FunctionTok{scale\_fill\_viridis\_c}\NormalTok{(}\AttributeTok{name =} \StringTok{"Predicted"}\NormalTok{, }\AttributeTok{option =} \StringTok{"plasma"}\NormalTok{, }\AttributeTok{limits =} \FunctionTok{c}\NormalTok{(}\DecValTok{0}\NormalTok{, }\DecValTok{15}\NormalTok{)) }\SpecialCharTok{+}
  \FunctionTok{labs}\NormalTok{(}\AttributeTok{title =} \StringTok{"KDE Baseline Predictions"}\NormalTok{) }\SpecialCharTok{+}
  \FunctionTok{theme\_crime}\NormalTok{()}

\NormalTok{p1 }\SpecialCharTok{+}\NormalTok{ p2 }\SpecialCharTok{+}\NormalTok{ p3 }\SpecialCharTok{+}
  \FunctionTok{plot\_annotation}\NormalTok{(}
    \AttributeTok{title =} \StringTok{"Actual vs. Predicted Burglaries"}\NormalTok{,}
    \AttributeTok{subtitle =} \StringTok{"Does our complex model outperform simple KDE?"}
\NormalTok{  )}
\end{Highlighting}
\end{Shaded}

\pandocbounded{\includegraphics[keepaspectratio]{assignment4_template_files/figure-pdf/compare-models-1.pdf}}

\begin{Shaded}
\begin{Highlighting}[]
\CommentTok{\# Calculate performance metrics}
\NormalTok{comparison }\OtherTok{\textless{}{-}}\NormalTok{ fishnet }\SpecialCharTok{\%\textgreater{}\%}
  \FunctionTok{st\_drop\_geometry}\NormalTok{() }\SpecialCharTok{\%\textgreater{}\%}
  \FunctionTok{filter}\NormalTok{(}\SpecialCharTok{!}\FunctionTok{is.na}\NormalTok{(prediction\_nb), }\SpecialCharTok{!}\FunctionTok{is.na}\NormalTok{(prediction\_kde)) }\SpecialCharTok{\%\textgreater{}\%}
  \FunctionTok{summarize}\NormalTok{(}
    \AttributeTok{model\_mae =} \FunctionTok{mean}\NormalTok{(}\FunctionTok{abs}\NormalTok{(countBurglaries }\SpecialCharTok{{-}}\NormalTok{ prediction\_nb)),}
    \AttributeTok{model\_rmse =} \FunctionTok{sqrt}\NormalTok{(}\FunctionTok{mean}\NormalTok{((countBurglaries }\SpecialCharTok{{-}}\NormalTok{ prediction\_nb)}\SpecialCharTok{\^{}}\DecValTok{2}\NormalTok{)),}
    \AttributeTok{kde\_mae =} \FunctionTok{mean}\NormalTok{(}\FunctionTok{abs}\NormalTok{(countBurglaries }\SpecialCharTok{{-}}\NormalTok{ prediction\_kde)),}
    \AttributeTok{kde\_rmse =} \FunctionTok{sqrt}\NormalTok{(}\FunctionTok{mean}\NormalTok{((countBurglaries }\SpecialCharTok{{-}}\NormalTok{ prediction\_kde)}\SpecialCharTok{\^{}}\DecValTok{2}\NormalTok{))}
\NormalTok{  )}

\NormalTok{comparison }\SpecialCharTok{\%\textgreater{}\%}
  \FunctionTok{pivot\_longer}\NormalTok{(}\FunctionTok{everything}\NormalTok{(), }\AttributeTok{names\_to =} \StringTok{"metric"}\NormalTok{, }\AttributeTok{values\_to =} \StringTok{"value"}\NormalTok{) }\SpecialCharTok{\%\textgreater{}\%}
  \FunctionTok{separate}\NormalTok{(metric, }\AttributeTok{into =} \FunctionTok{c}\NormalTok{(}\StringTok{"approach"}\NormalTok{, }\StringTok{"metric"}\NormalTok{), }\AttributeTok{sep =} \StringTok{"\_"}\NormalTok{) }\SpecialCharTok{\%\textgreater{}\%}
  \FunctionTok{pivot\_wider}\NormalTok{(}\AttributeTok{names\_from =}\NormalTok{ metric, }\AttributeTok{values\_from =}\NormalTok{ value) }\SpecialCharTok{\%\textgreater{}\%}
  \FunctionTok{kable}\NormalTok{(}
    \AttributeTok{digits =} \DecValTok{2}\NormalTok{,}
    \AttributeTok{caption =} \StringTok{"Model Performance Comparison"}
\NormalTok{  ) }\SpecialCharTok{\%\textgreater{}\%}
  \FunctionTok{kable\_styling}\NormalTok{(}\AttributeTok{bootstrap\_options =} \FunctionTok{c}\NormalTok{(}\StringTok{"striped"}\NormalTok{, }\StringTok{"hover"}\NormalTok{))}
\end{Highlighting}
\end{Shaded}

\begin{longtable}[t]{lrr}
\caption{\label{tab:model-comparison-metrics}Model Performance Comparison}\\
\toprule
approach & mae & rmse\\
\midrule
model & 2.48 & 3.59\\
kde & 2.06 & 2.95\\
\bottomrule
\end{longtable}

\subsubsection{8.3 \textbar{} Where Does the Model Work
Well?}\label{where-does-the-model-work-well}

\begin{Shaded}
\begin{Highlighting}[]
\CommentTok{\# Calculate errors}
\NormalTok{fishnet }\OtherTok{\textless{}{-}}\NormalTok{ fishnet }\SpecialCharTok{\%\textgreater{}\%}
  \FunctionTok{mutate}\NormalTok{(}
    \AttributeTok{error\_nb =}\NormalTok{ countBurglaries }\SpecialCharTok{{-}}\NormalTok{ prediction\_nb,}
    \AttributeTok{error\_kde =}\NormalTok{ countBurglaries }\SpecialCharTok{{-}}\NormalTok{ prediction\_kde,}
    \AttributeTok{abs\_error\_nb =} \FunctionTok{abs}\NormalTok{(error\_nb),}
    \AttributeTok{abs\_error\_kde =} \FunctionTok{abs}\NormalTok{(error\_kde)}
\NormalTok{  )}

\CommentTok{\# Map errors}
\NormalTok{p1 }\OtherTok{\textless{}{-}} \FunctionTok{ggplot}\NormalTok{() }\SpecialCharTok{+}
  \FunctionTok{geom\_sf}\NormalTok{(}\AttributeTok{data =}\NormalTok{ fishnet, }\FunctionTok{aes}\NormalTok{(}\AttributeTok{fill =}\NormalTok{ error\_nb), }\AttributeTok{color =} \ConstantTok{NA}\NormalTok{) }\SpecialCharTok{+}
  \FunctionTok{scale\_fill\_gradient2}\NormalTok{(}
    \AttributeTok{name =} \StringTok{"Error"}\NormalTok{,}
    \AttributeTok{low =} \StringTok{"\#2166ac"}\NormalTok{, }\AttributeTok{mid =} \StringTok{"white"}\NormalTok{, }\AttributeTok{high =} \StringTok{"\#b2182b"}\NormalTok{,}
    \AttributeTok{midpoint =} \DecValTok{0}\NormalTok{,}
    \AttributeTok{limits =} \FunctionTok{c}\NormalTok{(}\SpecialCharTok{{-}}\DecValTok{10}\NormalTok{, }\DecValTok{10}\NormalTok{)}
\NormalTok{  ) }\SpecialCharTok{+}
  \FunctionTok{labs}\NormalTok{(}\AttributeTok{title =} \StringTok{"Model Errors (Actual {-} Predicted)"}\NormalTok{) }\SpecialCharTok{+}
  \FunctionTok{theme\_crime}\NormalTok{()}

\NormalTok{p2 }\OtherTok{\textless{}{-}} \FunctionTok{ggplot}\NormalTok{() }\SpecialCharTok{+}
  \FunctionTok{geom\_sf}\NormalTok{(}\AttributeTok{data =}\NormalTok{ fishnet, }\FunctionTok{aes}\NormalTok{(}\AttributeTok{fill =}\NormalTok{ abs\_error\_nb), }\AttributeTok{color =} \ConstantTok{NA}\NormalTok{) }\SpecialCharTok{+}
  \FunctionTok{scale\_fill\_viridis\_c}\NormalTok{(}\AttributeTok{name =} \StringTok{"Abs. Error"}\NormalTok{, }\AttributeTok{option =} \StringTok{"magma"}\NormalTok{) }\SpecialCharTok{+}
  \FunctionTok{labs}\NormalTok{(}\AttributeTok{title =} \StringTok{"Absolute Model Errors"}\NormalTok{) }\SpecialCharTok{+}
  \FunctionTok{theme\_crime}\NormalTok{()}

\NormalTok{p1 }\SpecialCharTok{+}\NormalTok{ p2}
\end{Highlighting}
\end{Shaded}

\pandocbounded{\includegraphics[keepaspectratio]{assignment4_template_files/figure-pdf/prediction-errors-1.pdf}}

\subsubsection{8.4 \textbar{} Assess Model Performance ACross
Time}\label{assess-model-performance-across-time}

\section{Part 9: Summary Statistics}\label{part-9-summary-statistics}

\subsubsection{9.1 \textbar{} Model Summary
Table}\label{model-summary-table}

\begin{Shaded}
\begin{Highlighting}[]
\CommentTok{\# Create nice summary table}
\NormalTok{model\_summary }\OtherTok{\textless{}{-}}\NormalTok{ broom}\SpecialCharTok{::}\FunctionTok{tidy}\NormalTok{(final\_model, }\AttributeTok{exponentiate =} \ConstantTok{TRUE}\NormalTok{) }\SpecialCharTok{\%\textgreater{}\%}
  \FunctionTok{mutate}\NormalTok{(}
    \FunctionTok{across}\NormalTok{(}\FunctionTok{where}\NormalTok{(is.numeric), }\SpecialCharTok{\textasciitilde{}}\FunctionTok{round}\NormalTok{(., }\DecValTok{3}\NormalTok{))}
\NormalTok{  )}

\NormalTok{model\_summary }\SpecialCharTok{\%\textgreater{}\%}
  \FunctionTok{kable}\NormalTok{(}
    \AttributeTok{caption =} \StringTok{"Final Negative Binomial Model Coefficients (Exponentiated)"}\NormalTok{,}
    \AttributeTok{col.names =} \FunctionTok{c}\NormalTok{(}\StringTok{"Variable"}\NormalTok{, }\StringTok{"Rate Ratio"}\NormalTok{, }\StringTok{"Std. Error"}\NormalTok{, }\StringTok{"Z"}\NormalTok{, }\StringTok{"P{-}Value"}\NormalTok{)}
\NormalTok{  ) }\SpecialCharTok{\%\textgreater{}\%}
  \FunctionTok{kable\_styling}\NormalTok{(}\AttributeTok{bootstrap\_options =} \FunctionTok{c}\NormalTok{(}\StringTok{"striped"}\NormalTok{, }\StringTok{"hover"}\NormalTok{)) }\SpecialCharTok{\%\textgreater{}\%}
  \FunctionTok{footnote}\NormalTok{(}
    \AttributeTok{general =} \StringTok{"Rate ratios \textgreater{} 1 indicate positive association with burglary counts."}
\NormalTok{  )}
\end{Highlighting}
\end{Shaded}

\begin{longtable}[t]{lrrrr}
\caption{\label{tab:model-summary-table}Final Negative Binomial Model Coefficients (Exponentiated)}\\
\toprule
Variable & Rate Ratio & Std. Error & Z & P-Value\\
\midrule
(Intercept) & 8.108 & 0.077 & 27.016 & 0.000\\
abandoned\_cars & 0.998 & 0.002 & -0.959 & 0.337\\
abandoned\_cars.nn & 0.994 & 0.000 & -18.186 & 0.000\\
dist\_to\_hotspot & 1.000 & 0.000 & 0.621 & 0.535\\
\bottomrule
\multicolumn{5}{l}{\rule{0pt}{1em}\textit{Note: }}\\
\multicolumn{5}{l}{\rule{0pt}{1em}Rate ratios > 1 indicate positive association with burglary counts.}\\
\end{longtable}

\subsection{Key Findings Checklist}\label{key-findings-checklist}

\textbf{Technical Performance:}

\begin{itemize}
\tightlist
\item
  Cross-validation MAE: 2.7
\item
  Model vs.~KDE: {[}Which performed better?{]}
\item
  Most predictive variable: {[}Which had largest effect?{]}
\end{itemize}

\textbf{Spatial Patterns:}

\begin{itemize}
\tightlist
\item
  Burglaries are {[}evenly distributed / clustered{]}
\item
  Hot spots are located in {[}describe{]}
\item
  Model errors show {[}random / systematic{]} patterns
\end{itemize}

\textbf{Model Limitations:}

\begin{itemize}
\tightlist
\item
  Overdispersion: {[}Yes/No{]}
\item
  Spatial autocorrelation in residuals: {[}Test this!{]}
\item
  Cells with zero counts: {[}What \% of data?{]}
\end{itemize}




\end{document}
